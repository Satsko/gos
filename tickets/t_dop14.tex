\textbf{\LARGE dop 14. Задача Штуpма-Лиувилля и свойства ее pешений.}

Рассмотрим краевую задачу
\begin{equation}\tag{1}
    Ly = \frac{d}{dx} \Big( p(x) \frac{\partial y}{\partial x} \Big) - q(x) y = - \lambda y, \; 0\leq x \leq l
\end{equation}
\begin{equation}\tag{2}
    \alpha_1 y'(0)+\beta_1 y(0)=0
\end{equation}
\begin{equation}\tag{3}
    \alpha_2 y'(l)+\beta_2 y(l)=0
\end{equation}
где $p(x), q(x)$ - известные действительные функции, $\alpha_1, \alpha_2, \beta_1, \beta_2$ - известные действительные постоянные такие, что $p(x)\in C^1[0,l], p(x)>0, x\in[0,l], q(x)\in C[0,l], \alpha_i^2+\beta_i^2>0, i=1,2$ и $\lambda$ - комплексный параметр.

Очевидно, что при любом значении параметра $\lambda$ краевая задача (1-3) имеет решение $y(x) = 0$.

\textbf{Определение 1} Если для некоторого $\lambda_1$ краевая задача (1-3) имеет нетривиальное решение $y_1(x)$, то $\lambda_1$ называется собственным значением, а $y_1(x)$ --- собственной функцией.

Задача поиска собственных значений и собственных функций называется \textbf{задачей Штурма-Лиувилля.}

Собственные функции определены с точностью до произвольной постоянной: если $y(x)$ – собственная функция, то и $cy(x)$, где $c-const, \, c \neq 0$, является собственной функцией.

Задача решения уравнения (1) представляет собой задачу поиска собственных значений и собственных функций дифференциального
оператора $L$.

Собственные значения и собственные векторы действительной матрицы могут быть комплекснозначными, поэтому мы должны рассматривать комплекснозначные значения параметра $\lambda$ и комплекснозначные решения задачи (1-3).

\textbf{Свойства собственных функций и собственных значений задачи Штурма-Лиувилля.}

\textbf{Теорема 1} Все собственные функции и собственные значения
задачи Штурма-Лиувилля действительны.

\textbf{Доказательство.} Пусть $\lambda_1$ – собственное значение, а $y_1(x)$ – соответствующая ему собственная функция. Предположим, что они комплекснозначные, то есть $\lambda_1 = a + ib, y_1(x) = u(x) + iv(x)$. Так как функция
$y_1(x)$ является решением уравнения (1), то $Ly_1 = -\lambda_1 y_1(x)$. Записывая это равенство отдельно для действительных и мнимых частей,
получим
\begin{equation*}\tag{4}
    Lu=-au(x)+bv(x)
\end{equation*}
\begin{equation*}\tag{5}
    Lv=-bu(x)-av(x)
\end{equation*}
\\Так как функция $y_1(x)$ удовлетворяет краевым условиям (2), (3),
то и функции $u(x), v(x)$ удовлетворяют этим краевым условиям.
Умножим уравнение (4) на $v(x)$, а уравнение (5) на $u(x)$, проинтегрируем оба уравнения от 0 до $l$ и вычтем из первого второе.

В результате получим
\begin{equation*}
\int_{0}^{l}\big(v(x)Lu-u(x)Lv\big)dx = b\int_{0}^{l}\big(u^2(x)+v^2(x)\big)dx
\end{equation*}

Применяя следствие из теоремы Грина
\begin{equation*}
    \int_{0}^{l}\big(v(x)Lu-u(x)Lv\big)dx = 0
\end{equation*}
имеем
\begin{equation*}
b\int_{0}^{l}\big(u^2(x)+v^2(x)\big)dx = 0
\end{equation*}

Следовательно, $b=0$. Значит $\lambda_1$ действительно и $y_1(x)$ также действительно. $\blacksquare$

\textbf{Теорема 2.} Каждому собственному значению соответствует
только одна собственная функция.

\textbf{Доказательство.} Пусть собственному значению $\lambda$ соответствуют две
собственные функции $y_1(x), y_2(x)$. Это значит, что они являются решениями уравнения (1) и удовлетворяют краевым условиям (2),
(3). Из краевого условия (2) следует, что определитель Вронского $W [y_1, y_2](0) = 0$ Так как $y_1(x), y_2(x)$ – решения одного и того же линейного однородного дифференциального уравнения (1), то
$y_2(x) = cy_1(x)$. $\blacksquare$

Введем скалярное произведение функций $v(x)$ и $w(x)$
\begin{equation*}
(v,w)=\int_{0}^{l}v(x)w(x)dx
\end{equation*}
Будем называть функции $v(x)$ и $w(x)$ \textit{ортогональными}, если их скалярное произведение равно нулю, то есть $(v, w) = 0$

\textbf{Теорема 3}. Собственные функции, соответствующие различным собственным значениям, являются ортогональными.

\textbf{Доказательство.} Пусть $\lambda_1 \neq \lambda_2$ – различные собственные значения, а $y_1(x), y_2(x)$ – соответствующие им собственные функции. Так как $y_1(x), y_2(x)$ удовлетворяют краевым условиям (2), (3), то из следствия
из формулы Грина получим, что
\begin{equation*}
(Ly_1,y_2)-(y_1,Ly_2)= \int_{0}^{l}\big( Ly_1 y_2(x) - y_1(x) Ly_2 \big) dx = 0
\end{equation*}

Так как $Ly_1 = -\lambda_1y_1(x), Ly_2 = -\lambda_2y_2(x)$, то
\begin{gather*}
(\lambda_1-\lambda_2)(y_1,y_2)=\lambda_1(y_1,y_2)-\lambda_2(y_1,y_2)=(\lambda_1 y_1,y_2)-(y_1,\lambda_2 y_2)=\\
=-(Ly_1,y_2)+(y_1,Ly_2)=0
\end{gather*}
Следовательно, $(\lambda_1 - \lambda_2)(y_1, y_2) = 0$, а значит $(y_1, y_2) = 0$ и функции
$y_1(x), y_2(x)$ ортогональны. $\blacksquare$

\textbf{Теорема 4.} Пусть $\alpha_1 = \alpha_2 = 0$. Тогда, если $\lambda$ – собственное значение, то
\begin{equation*}\tag{6}
    \lambda \geq \min_{0 \leq x \leq l} q(x)
\end{equation*}

\textbf{Доказательство.} Предположим, что $\lambda_1$ - собственное значение, $y_1(x)$
- соответствующая собственная функция и
\begin{equation*}
\lambda_1 < \min_{0 \leq x \leq l} q(x)
\end{equation*}
\\Тогда $q(x)- \lambda_1 > 0$ на отрезке $[0, l]$. Из уравнения (1) следует, что
\begin{equation*}
\frac{d}{d x} \Big( p(x) \frac{d y_1}{d x} \Big) = (-\lambda_1+q(x))y_1(x)
\end{equation*}
Интегрируя от $0$ до $x$, получим
\begin{equation*}\tag{7}
    p(x)y_1'(x) = p(0)y_1'(0) + \int_{0}^{x}\big(q(s)-\lambda_1\big)y_1(s)ds
\end{equation*}
Так как $y_1(x)$ удовлетворяет краевым условиям (2), (3) и $\alpha_1 =
\alpha_2 = 0,$ то $y_1(0) = y_1(l) = 0$. Так как $y_1(x)$ – ненулевое решение (1), то
$y_1'(0) \neq 0$. Пусть для определенности $y_1'(0)>0$. Тогда $y_1'(x)>0$ при
$x \in [0, l]$. Предположим, что это не так. Обозначим через $x_0$ минимальное
число, при котором $y_1'(x_0) = 0$. Тогда для $x \in [0, x_0)$ производная $y_1'(x) >
0$, а значит и $y_1(x) > 0$ при $x \in (0, x_0)$. Положив в (7) $x = x_0$ и
учитывая положительность $q(x) - \lambda_1$, получим, что $y_1'(x_0) > 0$. Это
противоречие доказывает положительность $y_1'(x)$ при $x \in [0, l]$. Но тогда
$y_1(x) > 0$ при $x \in (0, l]$, что противоречит краевому условию $y_1(l) = 0$.
Следовательно, исходное предположение неверно и неравенство (6) доказано. $\blacksquare$

Рассмотрим собственные функции задачи Штурма-Лиувилля (1-3). Можно показать, что их счетное число. Следовательно все их можно занумеровать $y_n(x), \, n = 1, 2, \dots $. Чтобы устранить неопределенность, связанную с тем, что они содержат произвольный сомножитель,
будем считать, что
\begin{equation*}
    \int_0^l (y_n(x))^2 \, dx = 1.
\end{equation*}
Пусть $f(x)$ некоторая непрерывная на $[0, l]$ функция. Введем обозначение
\begin{equation*}
    f_n = \int \limits_0^l f(x) y_n(x) \, dx, \; n=1,2,\dots
\end{equation*}
\textbf{Теорема 5.} (Теорема Стеклова) Если $f(x) \in C^2[0, l]$ и удовлетворяет краевым условиям (2), (3), то ряд
\begin{equation*}
    \sum \limits_{n=1}^{\infty} f_n y_n(x)
\end{equation*}
сходится равномерно на отрезке $[0, l]$ к функции $f(x)$, то есть
\begin{equation*}
    f(x) = \sum \limits_{n=1}^{\infty} f_n y_n(x), \quad 0 \leq x \leq l.
\end{equation*}

% -------- source --------
\bigbreak
[\cite[page 67-73]{denisov2}]
