\textbf{\LARGE dop 26. Основные понятия теоpии pазностных схем: аппpоксимация, устойчивость, сходимость.}

Пусть дана исходная дифференциальная задача, которую мы запишем в виде
\begin{equation*}
    Lu(x)=f(x), \eqno(1)
\end{equation*}
где $x \in G$ , $G$ — область m-мерного пространства, $f(x)$ — заданная
функция, $L$ — линейный дифференциальный оператор. Предполагается, что дополнительные условия (типа начальных и граничных
условий) учтены оператором $L$ и правой частью $f$.

Для построения разностной схемы вводится \textbf{сетка}
$G_h$ — конечное множество точек, принадлежащих $G$, плотность распределения которых характеризуется параметром $h$ — \textbf{шагом сетки}.

В общем случае параметр $h$ — вектор, причем определена $|h|$ — длина вектора $h$. Обычно сетка $G_h$ выбирается так, что при
$|h| \rightarrow0$ множество $G_h$ стремится заполнить всю область $G$. Функция, определенная в точках сетки $G$, называется \textbf{сеточной функцией}.

После введения сетки $G_h$ следует заменить в уравнении (1) дифференциальный оператор $L$ разностным оператором $L_h$, правую
часть $f(x)$ — сеточной функцией $\varphi_n(x)$. В результате получим систему разностных уравнений
\begin{equation*}
    L_hy_h(x)=\varphi_h(x), \; x\in G_h, \eqno(2)
\end{equation*}
которая называется\textbf{ разностной схемой} или\textbf{ разностной задачей}.
% В отличие от дифференциального уравнения решение разностной
% задачи будем обозначать буквой $y$.

Заметим, что свойство аппроксимации означает близость разностного оператора к дифференциальному. Отсюда еще не следует, вообще говоря, близость
решений дифференциального и разностного уравнений. Свойство устойчивости разностной схемы является ее внутренним свойством, не зависящим от того, аппроксимирует ли эта схема какое-либо
дифференциальное уравнение. Оказывается, однако, что
если разностная схема аппроксимирует корректно поставленную задачу и устойчива, то ее решение сходится при $|h|\rightarrow 0$ к решению исходной дифференциальной задачи.

Будем считать, что решение $u(x)$ задачи (1) принадлежит линейному нормированному пространству $B_0, ||\cdot||_0$ —норма в $B_0$. Например, $B_0=C[a,b]$, $||u||_0 = \max_{x\in[a,b]} |u(x)|$. Аналогично считаем, что сеточные функции $y_h(x), \varphi_h(x)$ являются элементами линейного
нормированного пространства (пространства сеточных функций) $B_h$ с нормой $||\cdot||_h$. По существу, имеем семейство линейных нормированных пространств, зависящее от параметра h.

Чтобы иметь возможность сравнивать функции из различных пространств, вводится оператор проектирования $p_h: B_0 \rightarrow B_h$. Это,
по определению, линейный оператор, сопоставляющий каждой функции из $B_0$ некоторую функцию из $B_h$. Для функции $u\in B_0$ обозначим через $u_h$ ее проекцию на пространство $B_h$ т. е. $u_h(x) = p_h u(x)$.

В дальнейшем будем требовать, чтобы нормы в $B_h$ были согласованы с нормой в исходном пространстве $B_0$. Это означает, что для любой $u \in B_0$ выполняется условие
\begin{equation*}
    \lim_{|h| \rightarrow 0}||p_h u||_h=||u||_0 \eqno(3)
\end{equation*}

Требование согласования норм обеспечивает единственность предела сеточных функций при $|h|\rightarrow0$.

Пусть $u(x)$ — решение исходной задачи (1) и $y_h(x)$ — решение разностной задачи (2).

\textbf{Определение 1.} Сеточная функция $z_h(x) = y_h(x) - p_h u(x), x\in G_h$, называется \textit{погрешностью разностной схемы} (2).

Подставим $y_h(x) = p_hu(x) + z_h(x)$ в уравнение (2). Тогда получим, что погрешность $z_h(x)$ удовлетворяет уравнению
\begin{equation*}
    L_h z_h(x) = \psi_h(x), x\in G_h \eqno(4)
\end{equation*}
где
\begin{equation*}
    \psi_h(x) = \varphi_h(x) - L_h(p_h u(x))\equiv \varphi_h(x) - L_h u_h(x) \eqno(5)
\end{equation*}

\textbf{Определение 2.} Сеточная функция $\psi_h(x)$, определенная
формулой (5), называется \textit{погрешностью аппроксимации разностной задачи} (2) на решении исходной дифференциальной задачи (1).

Преобразуем выражение для $\psi_h(x)$. Проектируя уравнение (1) на сетку $G_h$, получим
\begin{equation}\nonumber
p_h Lu(x)=p_h f(x)
\end{equation}
или, учитывая принятые обозначения,
\begin{equation*}
(Lu)_h(x)=f_h(x) \eqno(6)
\end{equation*}
Из (5) и (6) получаем
\begin{equation}\nonumber
\psi_h(x)=\big[(Lu)_h(x)-L_hu_h(x)\big]+(\varphi_h(x)-f_h(x))
\end{equation}
т.е.
\begin{equation}\nonumber
\psi_h(x)=\psi_{h,1}(x) + \psi_{h,2}(x)
\end{equation}
где
\begin{equation*}
    \psi_{h,1}(x) = (Lu)_h(x)-L_hu_h(x),\quad \psi_{h,2}(x) = \varphi_h(x)-f_h(x) \eqno(7)
\end{equation*}
\textbf{Определение 3.} Функции $\psi_{h,1}(x)$ и $\psi_{h,2}(x)$ называются, соответственно, \textbf{погрешностью аппроксимации дифференциального
оператора} $L$ разностным оператором $L_h$ и \textbf{погрешностью аппроксимации правой части}.

\textbf{Определение 4.} Говорят, что разностная задача (2) \textbf{аппроксимирует} исходную задачу (1), если $||\psi_h||_h \rightarrow 0$ при $|h|\rightarrow0$. Разностная схема имеет $k$-й \textbf{порядок аппроксимации}, если существуют
постоянные $k>0, M_1>0$, не зависящие от $h$ и такие, что
\begin{equation}\nonumber
||\psi_h||_h \leq M_1|h|^k.
\end{equation}
Аналогично определяются погрешность аппроксимации и порядок погрешности аппроксимации правых частей и дифференциального оператора.

\textbf{Определение 5.} Разностная схема (2) называется \textbf{корректной}, если

1) ее решение существует и единственно при любых правых частях $\varphi_h \in B_h$

2) существует постоянная $M_2>0$, не зависящая от $h$ и такая, что при любых $\varphi_h \in B_h$ справедлива оценка
\begin{equation*}
    ||y||_h \leq M_2||\varphi_h||_h \eqno(8)
\end{equation*}

Свойство 2), означающее непрерывную зависимость, равномерную относительно $h$, решения разностной задачи от правой части, называется \textit{устойчивостью} разностной схемы. Заметим, что требование 1) эквивалентно существованию оператора $L_h^{-1}$, обратного оператору $L_h$, а требование 2) эквивалентно равномерной по $h$
ограниченности оператора $L_h^{-1}$.

\textbf{Определение 6.} Решение разностной задачи (2) \textbf{сходится} к решению дифференциальной задачи (1), если при $|h|\rightarrow0$
\begin{equation}\nonumber
||y_h - p_hu||_h \rightarrow 0.
\end{equation}

Разностная схема имеет $k$-й \textbf{порядок точности}, если существуют постоянные $k>0, M_3>0$, не зависящие от $h$ и такие, что
\begin{equation}\nonumber
||y_h-\rho_hu||_h \leq M_3|h|^k.
\end{equation}

\textbf{Теорема.} Пусть дифференциальная задача (1) поставлена корректно,
разностная схема (2) является корректной и аппроксимирует исходную задачу (1). Тогда решение разностной задачи (2) сходится
к решению исходной задачи (1), причем порядок точности совпадает с порядком аппроксимации.

\textbf{Доказательство.} Доказательство следует прямо из определений. Действительно, уравнение для погрешности (4) имеет ту же структуру, что
и разностная задача (2). Поэтому из требования корректности следует оценка
\begin{equation*}
    ||z_h||_h \leq M_2||\psi_h||_h \eqno(9)
\end{equation*}
Поскольку константа $M_2$ не зависит от $h$, получаем, что при $||\psi_h||_h \rightarrow 0$
норма погрешности $z_h$ также стремится к нулю, т. е. схема сходится. Если $||\psi_h||_h \leq M_1|h|^k$, то из (9) получим
\begin{equation}\nonumber
||z_h||_h \leq M_1M_2|h|^k
\end{equation}
т. е. разностная схема имеет k-й порядок точности. $\blacksquare$

Значение приведенной выше теоремы состоит в том, что она позволяет разделить изучение сходимости на два отдельных этапа: доказательство аппроксимации и доказательство устойчивости.
Обычно более сложным этапом является исследование устойчивости, которое состоит в получении оценок вида (8), называемых
\textbf{априорными оценками}.





% -------- source --------
\bigbreak
[\cite[page 286-291]{chm_samarski_gulin}]
