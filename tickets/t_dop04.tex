\textbf{\LARGE dop 4. Фоpмула Тейлоpа с остаточным членом в фоpме Лагpанжа. Разложение элементаpных функций.}

\textbf{Теорема Тейлора}

\mathLet Функция $f(x)$ имеет в некоторой окрестности  точки $a \ (n+1)$-ую производную $f^{(n+1)}(x)$. \mathLet \ $x$ --- значение из указанной окрестности, $p > 0$ - любое положительное число. Тогда между точками $a$ и $x$ найдется точка $\xi$ такая, что справедливо:

\begin{align}\label{Taylor}
    f(x) = f(a) + \frac{f'(a)}{1!}(x-a) + \frac{f^{(2)}(a)}{2!}(x-a)^2 + ... \notag\\
    ... + \frac{f^{(n)}(a)}{n!}(x-a)^n + R_{n+1}(x),
\end{align}
где
\begin{align}\label{R_n+1_condition}
    R_{n+1}(x) = \bigg(\frac{x-a}{x-\xi}\bigg)^p\frac{(x-\xi)^{n+1}}{n!p}f^{(n+1)}(\xi).
\end{align}
$R_{n+1}(x)$ - остаточный член в общей форме.

\begin{proof}
    Положим
    \begin{align}
        \phi(x,a) = f(a) + \frac{f'(a)}{1!}(x-a) + \frac{f^{(2)}(a)}{2!}(x-a)^2 + ... \notag \\
        ... + \frac{f^{(n)}(a)}{n!}(x-a)^n
    \end{align}
    Обозначим за $R_{n+1}(x)$ разность
    \begin{align}\label{R_n+1_in_proof}
        R_{n+1}(x) = f(x) - \phi(x,a)
    \end{align}
    Теорема будет доказана, если установим, что $R_{n+1}(x)$ определяется формулой (\ref{R_n+1_condition}).

    Фиксируем $\forall x \in$ окрестности, указанной в формулировке теоремы. Ради определенности \\ \mathLet $x>a$. Пусть $t$ --- переменная с областью изменения $[\,a, x]$, и рассмотрим функцию $\psi(t)$:
    \begin{align}\label{psi(t)}
        \psi(t) = f(x) - \phi(x,t) - (x-t)^pQ(x),
    \end{align}
    где
    \begin{align}\label{Q(x)}
        Q(x) = \frac{R_{n+1}(x)}{(x-a)^p}
    \end{align}
    т.e.
    \begin{align}\label{psi(t)_detail}
        \psi(t) = f(x) - f(t) - \frac{f'(t)}{1!}(x-t) - ... - \frac{f^{(n)}(t)}{n!}(x-t)^n - (x-t)^pQ(x)
    \end{align}
    Покажем, что $\psi(t)$ удовлетворяет на $[\,a,x]\,$ всем условиям теоремы Ролля.

    Из формулы (\ref{psi(t)_detail}) и из условий, наложенных на функцию $f(x)$ очевидно, что функция $\psi(t) \in C[\,a,x]\,$ и дифференцируема на $[\,a,x]\,$. При $t=a$ в (\ref{psi(t)}) и, учитывая (\ref{Q(x)}) получим
    \begin{align*}
        \psi(a) = f(x) - \phi(x,a) - R_{n+1}(x).
    \end{align*}

    Учитывая (\ref{R_n+1_in_proof}) получим $\psi(a) = 0$. А равенство $\psi(x) = 0$ сразу вытекает из (\ref{psi(t)_detail}).
    \bigbreak

    Итак, для $\psi(t)$ на  $[\,a,x]\,$ выполнены условия теоремы Ролля. Согласно этой теореме $\exists \xi \in (a,x): $
    \begin{align}\label{psi'(xi)}
        \psi'(\xi)=0
    \end{align}

    Подсчитаем $\psi'(t)$, продифференцировав (\ref{psi(t)_detail}):
    \begin{align}\label{psi'(t)}
        \psi'(t) = -f'(t) + \frac{f'(t)}{1!} - \frac{f^{(2)}(t)}{1!}(x-t) + \frac{f^{(2)}(t)}{2!}2(x-t) - ... \notag \\
        ... + \frac{f^n(t)}{n!}(x-t)^{n-1} - \frac{f^{(n+1)}(t)}{n!}(x-t)^n + p(x-t)^{p-1}Q(x)
      \end{align}

    Все члены в (\ref{psi'(t)}) кроме последних двух взаимно уничтожаются. Таким образом
    \begin{align}
        \psi'(t) = - \frac{f^{(n+1)}(t)}{n!}(x-t)^n + p(x-t)^{p-1}Q(x)
    \end{align}
    Полагая $t = \xi$ и, используя (\ref{psi'(xi)}), получим
    \begin{align}\label{Q(x)final}
        Q(x) = \frac{(x-\xi)^{n-p+1}}{n!p}f^{(n+1)}(\xi)
    \end{align}
    Из (\ref{Q(x)final}) и (\ref{Q(x)}) следует
    \begin{align*}
        R_{n+1}(x) = (x-a)^pQ(x) = \frac{(x-a)^p(x-\xi)^{n-p+1}}{n!p}f^{(n+1)}(\xi).
    \end{align*}
    Теорема доказана.
\end{proof}

\textbf{Остаточный член в форме Лагранжа} \\
Преобразуем формулу (\ref{R_n+1_condition}). Так как $\xi \in (a,x)$, то найдется такое $\theta \in (0,1): \ \xi -a = \theta(x-a)$. При этом $\xi = a + \theta(x-a), \ x-\xi = (x-a)(1-\theta)$. Тогда для из (\ref{R_n+1_condition}) получим:

\begin{align}\label{R_n+1_with_theta}
    R_{n+1}(x) = \frac{(x-a)^{n+1}(1-\theta)^{n-p+1}}{n!p} f^{(n+1)}[\,a + \theta(x-a)]\,
\end{align}

По условию теоремы Тейлора в качестве $p$ можно взять любое положительное число. Пусть $p = n + 1$.
Тогда из (\ref{R_n+1_with_theta}) получим остаточный член в \textbf{форме Лагранжа}:

\begin{align}
    R_{n+1}(x) = \frac{(x-a)^{n+1}}{(n+1)!}f^{(n+1)}[\,a + \theta(x-a)]\,.
\end{align}


\textbf{Разложение элементарных функций}\\
Взяв формулу Тейлора (\ref{Taylor}) при $a=0$ получим \textbf{формулу Маклорена}.

\begin{align}\label{Makloren}
    f(x) = f(0) + \frac{f'(0)}{1!}x + \frac{f^{(2)}(0)}{2!}x^2 + ... + \frac{f^{(n)}(0)}{n!}x^n + R_{n+1}(x),
\end{align}
где остаточный член в форме Лагранжа примет вид:
\begin{align*}
     R_{n+1}(x) = \frac{x^{n+1}}{(n+1)!}f^{(n+1)}[\,\theta x]\,.
\end{align*}

Разложения по формуле Маклорена:
\begin{enumerate}
    \item
        $
            e^x = 1 + \frac{x}{1!} + \frac{x^2}{2!} + ... + \frac{x^n}{n!} + R_{n+1}(x), \
        $

        $$
           \text{где} \ R_{n+1}(x) = \frac{x^{n+1}}{(n+1)!}e^{\theta x}, \quad
            \theta \in (0,1)
        $$

    \item
        $
            \sin x = x - \frac{x^3}{3!} + \frac{x^5}{5!} - \frac{x^7}{7!} + ... + (-1)^{n-1}\frac{x^{2n-1}}{(2n-1)!} + R_{2n+1}(x), \\
        $

        $$
            \text{где} \ R_{2n+1}(x) = \frac{x^{2n+1}}{(2n+1)!} \sin(\theta x + (2n+1) \frac{\pi}{2} + \pi), \quad \theta \in (0,1)
        $$
    \item
        $
            \cos x = 1 - \frac{x^2}{2!} + \frac{x^4}{4!} - \frac{x^6}{6!} + ... + (-1)^{n}\frac{x^{2n}}{(2n)!} + R_{2n+2}(x)
        $
        $$
            \text{где} \ R_{2n+2}(x) = \frac{x^{2n+2}}{(2n+2)!} \cos(\theta x + (2n+2) \frac{\pi}{2} + \pi), \quad \theta \in (0,1)
        $$
    \item
        $
            \ln(1+x) = x - \frac{x^2}{2} + \frac{x^3}{3} - \frac{x^4}{4} + ... + (-1)^{n-1}\frac{x^n}{n} + R_{n+1}(x),
        $
        $$
            \text{где} \ R_{n+1}(x) = \frac{(-1)^n x^{n+1}}{(n+1)(1+\theta x)^{n+1}}, \quad \theta \in (0,1)
        $$
    \item
        $
            (1+x)^\alpha = 1 + \frac{\alpha}{1!}x + \frac{\alpha (\alpha - 1)}{2!}x^2 + ... + \frac{\alpha(\alpha-1)...(\alpha - n + 1)}{n!}x^n + R_{n+1}(x),
        $
        $$
            \text{где} \ R_{n+1}(x) = \frac{\alpha(\alpha - 1)...(\alpha - n)}{(n+1)!}(1+\theta x)^{\alpha - (n+1)}x^{n+1}, \quad \theta \in (0,1)
        $$
\end{enumerate}


% -------- source --------
\bigbreak
[\cite{ilin_poznyak_matan}]
