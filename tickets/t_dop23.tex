\textbf{\LARGE dop 23. Теоpема о сходимости итеpационного метода для систем с симметpической положительно   опpеделенной матpицей.}

Пусть дана система уравнений $A x = f$,
где $A = [a_{ij}], \; i, j = 1,2,\ldots, m$ -- вещественная квадратная матрица, имеющая обратную, и $x = (x_1, x_2, \ldots, x_m)^T, \; f = (f_1, f_2, \ldots, f_m)^T$. \textbf{Канонической формой одношагового итерационного метода} называется его запись в виде 
$$
B_{n+1}\, \frac{x_{n+1} - x_{n}}{\tau_{n+1}} + A x_n = f, \;\; n = 0, 1, \ldots,
$$
где $n$ -- номер итерации, $x_0$ -- заданное начальное приближение, $x_n = (x_1^n, x_2^n, \ldots, x_m^n)^T$. Матрицы $B_{n+1}$ и числа $\tau_{n+1} > 0$ задают тот или иной конкретный итерационный метод.

Рассмотрим \textbf{стационарные одношаговые итерационные методы}
\begin{equation}
    B\, \frac{x_{n+1} - x_{n}}{\tau} + A x_n = f, 
    \tag{2}
\end{equation}

в которых матрица $B$ и числовой параметр $\tau$ не зависят от номера итерации $n$.

Будем рассматривать решение $x$ исходной системы  и последовательные приближения $x_n$ как элементы конечномерного линейного пространства $H$, а матрицы $A$, $B$ и другие -- как операторы, действующие в пространстве $H$. 
Предположим, что в $H$ введены скалярное произведение $(y, v)$ и норма $\|y\| = \sqrt{(y, y)}$. 
Для двух симметричных матриц $A$ и $B$ неравенство $A \geq B$ означает, что $\big(A x, x\big) \geq \big(B x, x\big)$ для всех $x \in H$. 
В случае симметричной положительно определенной матрицы $D$ будем обозначать $\|y\|_D = \sqrt{(D y , y)}$.\\

\textbf{Теорема:\;} Пусть $A$ и $B$ -- симметричные положительно определенные матрицы, для которых справедливы неравенства
\begin{equation}\tag{3}
    \gamma_1 \,B \leq A \leq \gamma_2 B,
\end{equation}

где $\gamma_1, \;\gamma_2$ -- положительные постоянные, $\gamma_2 > \gamma_1$. При 
$
\tau = \frac{2}{\gamma_1 + \gamma_2}
$ 
итерационный метод (2) сходится и для погрешности справедливы оценки 
$$
\|x_n - x\|_A \leq \rho^n \|x_0 - x\|_A, \;\; n = 0, 1, \ldots,
$$
$$
\|x_n - x\|_B \leq \rho^n \|x_0 - x\|_B, \;\; n = 0, 1, \ldots,
$$
где $\|v\|_A = \sqrt{(A v, v)}, \|v\|_B = \sqrt{(B v, v)}$\; и 
$
\rho = \frac{1 - \xi}{1 + \xi}, \quad \xi = \frac{\gamma_1}{\gamma_2}.
$

$\blacktriangleright\;$ Уравнение для погрешности $v_n = x_n - x$ имеет вид 
\begin{gather}
    \tag{4}
    B\, \frac{v_{n+1} - v_{n}}{\tau} + A v_n = 0, \;\; n = 0, 1, \ldots,\\
    \notag
    v_0 = x_0 - x,
\end{gather}
откуда получим 
\begin{equation}\tag{5}
    v_{n+1} = S v_n, \quad S = E - \tau B^{-1} A.
\end{equation}

\textbf{\quadЛемма 1:\;} Пусть $A$ и $B$ -- симметричные положительно определенные матрицы и $\rho >0$ -- число. Матричные неравенства 
\begin{equation}\tag{6}
    \frac{1-\rho}{\tau} \, B \leq A \leq \frac{1+\rho}{\tau} \, B
\end{equation}
необходимы и достаточны для того, чтобы при любых $v_0 \in H$ для решения задачи (4) выполнялась оценка 
\begin{equation} \tag{7}
    \|v_{n+1}\|_A \leq \rho \,\|v_n\|_A, \;\; n = 0, 1, \ldots\, .
\end{equation}

$\blacktriangleright\;$ Оценку (7) можно записать в виде 
\begin{equation}\tag{8}
    \|w_{n+1}\| \leq \rho\,\|w_n\|,
\end{equation}
где $w_n = A^{1/2} v_n,\; \|w_n\| = \sqrt{(w_n, w_n)}$. Из (5) получим, что функция $w_n$ удовлетворяет уравнению 
$
w_{n+1} = \widetilde{S} w_n,
$ 
где $\widetilde{S} = A^{1/2}S\,A^{-1/2} = E - \tau\,C, \quad C = A^{1/2}B^{-1}A^{1/2}$. Для решения этого уравнения в силу симметричности матрицы $\widetilde{S}$ имеем 
$$
\|w_{n+1}\|^2 = (\widetilde{S} w_n, \widetilde{S} w_n) = (\widetilde{S}^2\,w_n, w_n).
$$
Тем самым оценка (8) эквивалентна неравенству 
\begin{equation}
    \tag{9}
    \widetilde{S}^2 \leq \rho^2 \,E
\end{equation}
и остается доказать эквивалентность неравенств (6) и (9).

Справедливо следующее свойство -- для симметричной матрицы $S$ и любого числа $\rho >0$ эквивалентны следующие матричные неравенства: 
$
-\rho\, E \leq S \leq \rho\,E \quad \textit{и}  \quad S^2 \leq \rho^2\,E.
$

Тогда неравенство (9) эквивалентно двум матричным неравенствам  $\;-\rho\,E \leq \widetilde{S} \leq \rho \,E
\;$ или 
$$
\frac{1-\rho}{\tau}\, E \leq C\leq \frac{1+\rho}{\tau}\,E.
$$
Так как $ C = A^{1/2}B^{-1}A^{1/2}$ -- симметричная положительно определенная матрица, то следующие неравенства эквивалентны: $\;
\alpha\, C \geq \beta\, E \quad \textit{и} \quad \alpha \, E \geq \beta \, C^{-1},$ 
где $\alpha, \; \beta$ -- любые действительные числа. Следовательно, можем перейти к обратным матрицам:
$$
\frac{1-\rho}{\tau}\, C^{-1} \leq E\leq \frac{1+\rho}{\tau}\,C^{-1}.
$$
Подставляя сюда выражение для $C$, получим
$$
\frac{1-\rho}{\tau}\, A^{-1/2}B\,A^{-1/2} \leq E\leq \frac{1+\rho}{\tau}\, A^{-1/2}B\,A^{-1/2}.
$$
Справедливо следующее свойство -- пусть $ A^T = A$ и $L$ -- невырожденная матрица, тогда эквивалентны неравенства 
$
\;A \geq 0\quad \textit{и} \quad L^T A L \geq 0.$ Тогда умножив справа и слева последние полученные неравенства на $A^{
1/2}$, получим неравенства (6). $\;\blacksquare$

\textbf{\quadЛемма 2:\;} При тех же условиях, что и в лемме 1, неравенства (6) необходимы и достаточны для выполнения оценки 
$$
\|v_{n+1}\|_B \leq \rho\,\|v_n\|_B,\;\; n = 0, 1, \ldots \,.
$$
$\blacktriangleright\;$ Доказательство проводится аналогично лемме 1, только в качестве вектора $w$ надо взять вектор $B^{1/2} v_n$, а в качестве $C$ -- матрицу $B^{-1/2} A\, B^{-1/2}$. $\;\blacksquare$.\\

Для доказательства теоремы теперь достаточно заметить, что матричные неравенства (3) можно переписать в виде (6), где $
\rho = \frac{1 - \xi}{1+\xi}, \quad \xi = \frac{\gamma_1}{\gamma_2}, \quad \tau=\frac{2}{\gamma_1 + \gamma_2}.
$
После этого замечания утверждение теоремы  следует из лемм 1 и 2. $\;\blacksquare$\\

\textbf{Следствия из теоремы о сходимости}\\
Рассмотрим обобщенную задачу на собственные значения 
\begin{equation}
    \tag{10}
    A\, \mu = \lambda B \mu. 
\end{equation}
Если для матриц $A$ и $B$ выполнены неравенства (3), то из (10) для любого собственного вектора получим неравенства 
$$
\gamma_1 \big(B \mu, \mu) \leq \big( A \mu, \mu \big) = \lambda \big(B \mu, \mu\big) \leq \gamma_2 \big(B \mu, \mu\big).$$
Отсюда следует, что 
\begin{equation}
    \tag{11}
    \gamma_1 \leq \lambda_{min}(B^{-1}A), \quad \gamma_2 \geq \lambda_{max}(B^{-1}A),
\end{equation}
где $\lambda_{min}(B^{-1}A), \;  \lambda_{max}(B^{-1}A) $ -- минимальное и максимальное собственные числа задачи (10).

Таким образом, наиболее точными константами, с которыми выполняются неравенства (3), являются константы $\gamma_1 = \lambda_{min}(B^{-1}A), \quad \gamma_2 = \lambda_{max}(B^{-1}A)$. В этом случае параметр 
$$
\tau_0 = \frac{2}{\lambda_{min}(B^{-1}A) + \lambda_{max}(B^{-1}A)}
$$
называется \textbf{оптимальным итерационным параметром}, так как он минимизирует величину $\rho$ на множестве всех положительных $\gamma_1,\, \gamma_2$, удовлетворяющих условиям (11).\\

\textbf{Следствие 1:\;} Если $A^T = A > 0$, то для \textbf{метода простой итерации }
\begin{equation}
    \tag{12}
    \frac{x_{n+1} - x_{n}}{\tau} + A x_n = f
\end{equation}
при $\tau = \tau_0 = \frac{2}{\lambda_{min}(A) + \lambda_{max}(A)}$ справедлива оценка 
$\|x_n - x \| \leq \rho_0\,\|x_0 -x\|$, где $\rho_0 = \frac{1-\xi}{1+\xi}, \; \xi = \frac{\lambda_{min}(A)}{\lambda_{max}(A)}$.\\

\textbf{Следствие 2:\;} Для симметричной матрицы $A$ справедливо равенство $\|E - \tau_0\, A\| = \rho_0$.\\

% -------- source --------
\bigbreak
[\cite[page 90, 96-98, 100-102]{chm_samarski_gulin}]