\textbf{\LARGE dop 22. Пpимеpы и канонический вид одношаговых итеpационных методов pешения систем линейных алгебpаических уpавнений.}
\renewcommand{\theequation}{\arabic{equation}}

\textbf{I. Итерационные методы Якоби и Зейделя}

\par Рассмотрим систему $Ax=f$, где матрица $A=\begin{bmatrix}a_{ij}\end{bmatrix}$, $i$, $j=1$, $2$,..., $m$, имеет обратную, $x=\left( x_{1},\ldots ,x_{m}\right) ^{T}$, $f=\left( f_{1},\ldots ,f_{m}\right) ^{T}$.
\par Рассмотрим примеры итерационных методов. Для их построения преобразуем $Ax=f$ к виду 
\begin{equation} \label{eq22.1}
    x_{i}=-\sum ^{i-1}_{j=1}\dfrac{a_{ij}}{a_{ii}}x_{j}-\sum ^{m}_{j=i+1}\dfrac{a_{ij}}{a_{ii}}x_{j}+\dfrac{f_{i}}{a_{ii}},\;\; i=1, 2,\ldots, m, \;\;a_{ii}\neq 0. 
\end{equation}
Пусть значение суммы равно нулю, если верхний предел суммирования меньше нижнего. Тогда при $i=1$ уравнение \eqref{eq22.1} имеет вид: 
$$x_{1}=-\sum ^{m}_{j=2}\dfrac{a_{1j}}{a_{11}}x_{j}+\dfrac{f_{1}}{a_{11}}.$$ 
В дальнейшем верхний индекс это номер итерации, например: $x^{n}=\left( x_{1}^{n},\ldots ,x_{m}^{n}\right) ^{T}$, где $x_{i}^{n}$- $n$-ая итерация $i$-ой компоненты $\overrightarrow{x}$.\\
\par В \textbf{методе Якоби} исходят из записи системы в виде \eqref{eq22.1}, а итерации определяются следующим образом:
\begin{equation}\label{eq22.5}
   x_{i}^{n+1}=-\sum ^{i-1}_{j=1}\dfrac{a_{ij}}{a_{ii}}x_{j}^{n}-\sum ^{m}_{j=i+1}\dfrac{a_{ij}}{a_{ii}}x_{j}^{n}+\dfrac{f_{i}}{a_{ii}},
\end{equation}
где $n=0$, $1$,..., $n_{0}$, $i=1$, $2$,..., $m$. Начальные значения $x_{i}^{0}$, $i=1$, $2$,..., $m$ задаются произвольно. Окончание итераций определяется либо заданием максимального числа итераций $n_{0}$, либо условием: $\max_{1\leq i\leq m} \left| x_{i}^{n+1}-x_{i}^{n}\right| <\varepsilon$, где $\varepsilon>0$ -- заданное число.\\
\par\textbf{Итерационный метод Зейделя} имеет вид: 
\begin{equation}\label{eq22.2}
    x_{i}^{n+1}=-\sum ^{i-1}_{j=1}\dfrac{a_{ij}}{a_{ii}}x_{j}^{n+1}-\sum ^{m}_{j=i+1}\dfrac{a_{ij}}{a_{ii}}x_{j}^{n}+\dfrac{f_{i}}{a_{ii}},
\end{equation}
где $n=0$, $1$,..., $n_{0}$, $i=1$, $2$,..., $m$. 
Распишем подробнее первые два уравнения системы \eqref{eq22.2}: 
\begin{align}
\label{eq22.3}
    x_{1}^{n+1} &=-\sum ^{m}_{j=2}\dfrac{a_{1j}}{a_{11}}x_{j}^{n}+\dfrac{f_{1}}{a_{11}},\\
\label{eq22.4}
x_{2}^{n+1}&=-\dfrac{a_{21}}{a_{22}}x_{1}^{n+1}-\sum ^{m}_{j=3}\dfrac{a_{2j}}{a_{22}}x_{j}^{n}+\dfrac{f_{2}}{a_{22}}.
\end{align}
Первая компонента $x_{1}^{n+1}$ вектора $x^{n+1}$ находится из уравнения \eqref{eq22.3} явным образом, для ее вычисления нужно знать $x^{n}$ и $f_1$. 
При нахождении $x_{2}^{n+1}$ из \eqref{eq22.4} используются найденное значение $x_{1}^{n+1}$ и известные значения $x_{j}^{n}$, $j=3$,..., $m$, с предыдущей итерации. Таким образом, компоненты $x_{i}^{n+1}$ вектора $x^{n+1}$ находятся из уравнения \eqref{eq22.2} последовательно, начиная с $i=1$.\\

\textbf{II. Матричная запись методов Якоби и Зейделя}

\par Для исследования сходимости удобнее записывать методы в матричной форме. 
Представим матрицу $A$ системы $Ax=f$ в виде суммы трех матриц 
$$A=A_{1}+D+A_{2},$$ 
где $D=\operatorname{diag}\left[ a_{11},a_{22},\ldots,a_{mm}\right]$ -- диагональная матрица с той же главной диагональю, что и матрица $A$, $A_{1}$ -- нижняя треугольная и $A_{2}$ -- верхняя треугольная с нулевыми главными диагоналями. 

Представление $Ax=f$ в форме \eqref{eq22.1} эквивалентно ее записи в виде матричного уравнения: 
$$x=-D^{-1}A_{1}x-D^{-1}A_{2}x+D^{-1}f.$$
\par \textbf{Метод Якоби} \eqref{eq22.5} в векторной записи: 
\begin{equation*}
   x^{n+1}=-D^{-1}A_{1}x^{n}-D^{-1}A_{2}x^{n}+D^{-1}f 
\end{equation*}
%x^{n+1}=-D^{-1}A_{1}x^{n}-D^{-1}A_{2}x^{n}+D^{-1}f$ 
или 
\begin{equation} \label{eq22.6}
    Dx^{n+1}+\left( A_{1}+A_{2}\right)x^{n}=f
\end{equation}
\par \textbf{Метод Зейделя }\eqref{eq22.2} в векторной форме: 
\begin{equation*}
    x^{n+1}=-D^{-1}A_{1}x^{n+1}-D^{-1}A_{2}x^{n}+D^{-1}f
\end{equation*} 
или 
\begin{equation}\label{eq22.7}
    \left( D+A_{1}\right)x^{n+1}+A_{2}x^{n}=f
\end{equation}
\par Учитывая, что $A = A_1 + D + A_2$, методы \eqref{eq22.6}, \eqref{eq22.7} можно переписать соответственно в виде: 
\begin{align}
\label{eq22.8}
D\left( x^{n+1}-x^{n}\right)+Ax^n &= f,\\
\label{eq22.9}
\left( D+A_{1}\right)\left( x^{n+1}-x^{n}\right)+Ax^n&=f. 
\end{align}
\par Видно, что если итерационный метод сходится, то он сходится к решению исходной системы уравнений. Для ускорения сходимости вводят числовые параметры, которые зависят от номера итерации. Например, в методы \eqref{eq22.8}, \eqref{eq22.9} можно ввести \textbf{итерационные параметры} $\tau_{n+1}$:
\begin{align*}
    D\frac{\left( x^{n+1}-x^{n}\right)}{\tau _{n+1}}+Ax^{n}&=f,\\
    \left( D+A_{1}\right)\frac{\left( x^{n+1}-x^{n}\right)}{\tau _{n+1}}+Ax^{n}&=f.
\end{align*}
\par Методы Якоби и Зейделя относятся к \textbf{одношаговым итерационным методам}, когда для нахождения $x^{n+1}$ требуется помнить только одну предыдущую итерацию $x^{n}$. Используются и многошаговые итерационные методы, в которых $x^{n+1}$ определяется через значения на двух и более предыдущих итерациях.\\

\textbf{III. Каноническая форма одношаговых методов}

\par Теперь пусть $x_{n}$ будет обозначать вектор, полученный в результате $n$-ой итерации. 
\par \fbox{\textbf{Опр.}} \textit{Канонической формой одношагового итерационного метода} решения $Ax=f$ называется его запись в виде: 
\begin{equation} \label{eq22.10}
    B_{n+1}\frac{\left( x_{n+1}-x_{n}\right)}{\tau _{n+1}}+Ax_{n}=f, \quad n = 0, 1, \ldots, n_0.
\end{equation}
Здесь $B_{n+1}$ — матрица, задающая итерационный метод, $\tau _{n+1}$— итерационный параметр. Предполагается, что задано начальное приближение $x_{0}$ и $\exists$ матрицы $B_{n}^{-1}$, $n=1$,..., $n_{0}-1$ . Тогда из \eqref{eq22.10} можно последовательно определить все $x_{n}$, $n=1$,..., $n_{0}$. Для нахождения $x_{n+1}$ по известным $f$ и $x_{n}$ достаточно решить систему уравнений: 
$$B_{n+1}x_{n+1}=F_{n},$$ 
где $F_{n}=\left( B_{n+1}-\tau_{n+1}A\right)x_{n}+\tau_{n+1}f$. \\

\par \fbox{\textbf{Опр.}} Итерационный метод называют \textit{явным (неявным)}, если $B_{n}=E$
($B_{n}\neq E$), где $E$— единичная матрица. \\

Неявные применяют, когда каждую $B_{n}$ обратить легче, чем исходную матрицу $A$. Например, в методе Зейделя приходится обращать треугольную матрицу.\\

\par \fbox{\textbf{Опр.}} Итерационный метод \eqref{eq22.10} называется \textit{стационарным}, если $B_{n+1}=B$ и $\tau_{n+1}=\tau$ не зависят от номера итерации, и \textit{нестационарным} в противоположном случае.\\

\textbf{IV. Другие примеры итерационных методов}

\par 1) \textbf{Методом простой итерации} называют явный метод с постоянным параметром $\tau$:
$$\frac{\left( x_{n+1}-x_{n}\right)}{\tau}+Ax_{n}=f$$ 
\par 2) Явный метод с переменным параметром $\tau_{n+1}$ называется \textbf{итерационным методом Ричардсона}:
$$\frac{\left( x_{n+1}-x_{n}\right)}{\tau_{n+1}}+Ax_{n}=f$$ 
\par 3) Обобщением метода Зейделя \eqref{eq22.9} является \textbf{метод верхней релаксации}: 
$$\left( D+\omega A_{1}\right)\frac{\left( x_{n+1}-x_{n}\right)}{\omega}+Ax_n=f,$$ 
где $\omega>0$ это заданный числовой параметр.

% -------- source --------
\bigbreak
[\cite[pages 82-85]{chm_samarski_gulin}]
