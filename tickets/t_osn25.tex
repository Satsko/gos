\textbf{\LARGE osn 25. Линейные обыкновенные дифференциальные уравнения и системы. Фундаментальная система решений. Определитель Вронского.}



\textbf{Линейным дифференциальным оператором} $n$-го порядка называется оператор $\mathcal{L}$, \textbf{линейным лифференциальным уравнением}  $n$-го порядка называется $f(t)$:
$$\mathcal{L} y = f(t) = a_0(t)y^{(n)}(t) + a_1(t)y^{(n-1)}(t) +\dots + a_{n-1}(t)y'(t) + a_n(t)y(t),$$

где $\mathcal{L}y(t) \in C[a, b]$, $f(t)$ -- комплекснозначная функция, 

$a_k(t) \in C[a,b], a_k(t) \in R$ $a_0(t) \neq 0, t \in [a,b]$,

$y(t) \in C^{(n)}[a,b]$ ($n$ раз диф-ма на отрезке $[a,b]$), 


Если $f(t) = 0$ на $[a, b]$, то уравнение называется \textbf{однородным}, иначе \textbf{неоднородным}.

\textbf{Теорема 1.}:
Если функции $y_k(t), k=1..m$ являются решениями уравнений $\mathcal{L} y_k = f_k(t)$, то функция $y(t) = \sum_{k=1}^m c_k y_k(t)$, где $c_k$ -- комплексные постоянные, -- является решением уравнения  $\mathcal{L} y = f(t)$, где $f(t) =  \sum_{k=1}^m c_k f_k(t)$.

\begin{proof}
$\mathcal{L} y = \mathcal{L} \sum_{k=1}^m c_k y_k(t) = \sum_{k=1}^m c_k \mathcal{L} y_k(t) = \sum_{k=1}^m c_k f_k(t) = f(t), \ t \in [a,b]$
\end{proof}

\textbf{Следствие}: Линейная комбинация решений однородного уравнения является решением однородного уравнения. Разность двух решений неоднородного уравнения с одинаковой правой частью есть решение однородного уравнения

\textbf{Теорема 2} Решение задачи Коши $\mathcal{L}y = f(t),~y^{(k)}(t_0) = y_{0k},~k = \overline{0, n - 1}$ представимо в виде $y(t) = v(t) + w(t)$, где функция $v(t)$ является решением задачи Коши для \textit{неоднородного} уравнения $\mathcal{L}v = f(t)$ с нулевыми начальными условиями $v^{(k)}(t_0) = 0, k = \overline{0, n - 1}$, а функция $w(t)$ является решением задачи Коши для \textit{однородного} уравнения $\mathcal{L}w = 0$ с ненулевыми начальными условиями $w^{(k)}(t_0) = y_{0k},~k = \overline{0, n - 1}$.

\begin{proof}
Сумма $y(t) = v(t) + w(t)$ удовлетворяет неоднородому ур-ю в ситу теоремы 1. Для начальных условий имеем равенства $y^{(k)}(t_0) = v^{(k)}(t_0) + w^{(k)}(t_0) = 0 + y_{0k} = y_{0k}, \ k=1..n-1$
\end{proof}

Скалярные функции $\varphi_1(t),\dots ,\varphi_m(t)$ называются \textbf{линейно зависимыми} на отрезке $[a,b]$, если найдутся такие комплексные константы 
$c_k \in \mathbb{C},~k = \overline{1,m}, \sum_{k=1}^m |c_k| > 0$, 
что справедливо равенство
$\displaystyle \sum_{k=0}^m c_k\varphi_k(t) = 0,~\forall t \in [a, b].$
Если равенство выполнено только для $c_k = 0, \ k=1..n$, то функции \textbf{линейно независимы}.

\textbf{Определителем Вронского} (вронскианом) системы функций $\varphi_1(t),\dots, \varphi_m(t)$, 
где $\varphi_i(t) \in C^{(m-1)}[a, b]$, называется зависящий от переменной $t \in [a, b]$ определитель
$$W[\varphi_1,\dots,\varphi_m](t) = \begin{vmatrix} \varphi_1(t) & \varphi_2(t) & \dots & \varphi_m(t) \\ \varphi'_1(t) & \varphi'_2(t) & \dots & \varphi'_m(t) \\ \vdots & \vdots & \ddots & \vdots \\ \varphi^{(m-1)}_1(t) & \varphi^{(m-1)}_2(t) & \dots & \varphi^{(m-1)}_m(t) \\ \end{vmatrix}$$


\textbf{Теорема 3}: если система скалярных функций $\varphi_1(t),\dots, \varphi_m(t)$, 
где $\varphi_i(t) \in C^{(m-1)}[a, b]$,
является линейно зависимой на отрезке $[a, b]$, то определитель Вронского этой системы тождественно равен нулю на этом отрезке: $W[\varphi_1,\dots ,\varphi_m](t) \equiv 0, \forall t \in [a,b]$.

\begin{proof}
Так как функции $\varphi_k(t)$ линейно зависимы на $[a,b]$, то существует нетривиальный набор констант $c_1,\dots, c_m$, для которого на отрезке $[a,b]$ справедливо равенство выше. В этом равенстве допустимо почленное дифференцирование до порядка $m - 1$ включительно:
$$c_1\varphi^{(k)}_1(t)+\dots +c_m\varphi^{(k)}_m(t)=0,~k=\overline{0,m-1},~t\in[a,b].$$
Отсюда следует, что вектор-столбцы определителя Вронского линейно зависимы для всех $t \in [a,b]$. Следовательно, этот определитель равен нулю для всех $t \in [a, b]$.
\end{proof}

% \textbf{Замечание.} Из ЛНЗ не следует, что $W\neq 0$, контрпример: $\varphi_1(t) = t^3,~\varphi_2(t) = t^2,~\varphi_3(t) = |t|,~W \equiv 0$, но на $[-1, 1]$ функции ЛНЗ.

\textbf{Замечание.} Из ЛНЗ не следует, что $W\neq 0$, контрпример: $\varphi_1(t) = t^2,~\varphi_2(t) = \{-t^2, t < 0; t^2, t \geq 0\}, W \equiv 0$, но на $[-1, 1]$ функции ЛНЗ.

\textbf{Теорема 4}: Для решений $y_1(t),\dots, y_n(t)$ линейного однородного уравнения на отрезке $[a, b]$ справедлива следующая \textbf{альтернатива}: 

либо $W[y_1,\dots, y_n](t) = 0$ на $[a, b]$ и функции $y_1(t),\dots ,y_n(t)$ линейно зависимы на этом отрезке;

либо $W[y_1,\dots ,y_n](t)\neq 0, \forall t \in [a, b]$ и функции $y_1(t),\dots , y_n(t)$ линейно независимы на $[a, b]$.

\textbf{Фундаментальной системой решений} линейного однородного дифференциального уравнения $n$-го порядка на отрезке $[a, b]$ называется система из $n$ линейно независимых на данном отрезке решений этого уравнения.

\textbf{Общим решением} линейного однородного (неоднородного) дифференциального уравнения $n$-го порядка называется зависящее от $n$ произвольных постоянных решение этого уравнения такое, что любое другое решение уравнения может быть получено из него в результате выбора некоторых значений этих постоянных.

\textbf{Теорема 5}: У любого линейного однородного уравнения $\mathcal{L}y = 0$ существует фундаментальная система решений на $[a, b]$.

\begin{proof}
Рассмотрим постоянную матрицу $B$ с элементами $b_{ij}$, $i,j = 1, 2,\dots , n$ такую, что $det B\neq  0$. Обозначим через $y_j(t)$ решения задачи Коши для уравнения $\mathcal{L} y = 0$ с начальными условиями:
$$y_j(t_0) = b_{1j},~y'(t_0) = b_{2j},\dots ,~y_{n-1}(t_0) = b_{nj}, j = 1,2,\dots ,n.$$
По теореме существования и единственности решения задачи Коши для линейного дифференциального уравнения $n$-го порядка функции $y_j(t)$ существуют и определены однозначно. Составленный из них определитель Вронского $W[y_1,\dots , y_n](t)$, в силу начальных условий, таков, что
$W[y_1,\dots ,y_n](t_0) = detB\neq 0$. Следовательно, по предыдущей теореме он не равен нулю ни в одной точке отрезка $[a, b]$. Значит, они образуют фундаментальную систему решений уравнения $\mathcal{L}y = 0$.
\end{proof}

\textbf{Теорема 6}: Пусть $y_1(t), y_2(t),\dots, y_n(t)$ --- фундаментальная система решений линейного однородного уравнения уравнения $\mathcal{L}y = 0$ на отрезке $[a, b]$. Тогда общее решение этого уравнения на рассматриваемом отрезке имеет вид:
\begin{equation}
    y_{OO}(t) = c_1y_1(t) + c_2y_2(t) +\dots + c_ny_n(t),~\forall c_j \in \mathbb{C}.
    \label{t2}
\end{equation}

\begin{proof}
Так как линейная комбинация решений однородного уравнения $\mathcal{L}y = 0$ является решением этого уравнения, то при любых значениях постоянных $c_k$ функция $y_{OO}(t)$, определяемая формулой (\ref{t2}), является решением линейного однородного дифференциального уравнения $\mathcal{L}y = 0$.

Покажем теперь, что любое решение уравнения $\mathcal{L}y = 0$ может быть получено из (\ref{t2}) в результате выбора значений постоянных $c_k$. Пусть $\tilde{y}(t)$ --- некоторое решение уравнения $\mathcal{L}y = 0$. Рассмотрим систему алгебраических уравнений относительно неизвестных $c_k$:
\begin{equation}
    \begin{cases}
        c_1y_1(t_0) + c_2y_2(t_0) +\dots + c_ny_n(t_0) = \tilde{y}(t_0)&\\
        c_1y_1'(t_0) + c_2y_2'(t_0) +\dots + c_ny_n'(t_0) = \tilde{y}'(t_0)&\\
        \dots&\\
        c_1y_1^{(n-1)}(t_0) + c_2y_2^{(n-1)}(t_0) +\dots + c_ny_n^{(n-1)}(t_0) = \tilde{y}^{(n-1)}(t_0)&\\
    \end{cases}
    \label{t2_2}
\end{equation}
где $t_0$ --- некоторая точка отрезка $[a, b]$. Определитель этой системы равен определителю Вронского в точке $t_0$ и не равен 0, так как решения $y_1(t), y_2(t),\dots , y_n(t)$ линейно независимы. Следовательно, система (\ref{t2_2}) имеет единственное решение $\tilde{c}_1, \tilde{c}_2 ,\dots, \tilde{c}_n$.

Рассмотрим функцию $\hat{y}(t) = \displaystyle\sum_{k=1}^{n}\tilde{c}_ky_k(t).$

Эта функция является решением уравнения $\mathcal{L}y = 0$. Так как постоянные $\tilde{c}_1, \tilde{c}_2 ,\dots, \tilde{c}_n$ представляют собой решение системы (\ref{t2_2}), то функция $\hat{y}(t)$ такова, что $\hat{y}^{(k)}(t_0)=\tilde{y}^{(k)}(t_0),~k=0,1,\dots ,n-1$.

Следовательно, функции $\hat{y}(t)$ и $\tilde{y}(t)$ являются решениями уравнения $\mathcal{L}y = 0$ и удовлетворяют одним и тем же начальным условиям в точке $t0$. По теореме о существовании и единственности решения задачи Коши эти функции должны совпадать: $\hat{y}(t)=\tilde{y}(t)=\displaystyle\sum_{k=1}^{n}\tilde{c}_ky_k(t)$
\end{proof}

\textbf{Теорема 7}: Пусть $y_1(t), y_2(t),\dots, y_n(t)$ --- фундаментальная система решений линейного однородного уравнения $\mathcal{L}y = 0$ на отрезке $[a, b]$, $y_H(t)$ --- некоторое (частное) решение неоднородного уравнения $\mathcal{L}y = f(t)$. Тогда общее решение линейного неоднородного уравнения $\mathcal{L}y = f(t)$ на рассматриваемом отрезке имеет вид:
\begin{equation}
    y_{OH}(t) = y_{H}(t) + y_{OO}(t) = y_H(t) + c_1y_1(t) + c_2y_2(t) + \dots  + c_ny_n(t),
    \label{t3}
\end{equation}
где $c_1,c_2,\dots,c_n$ --- произвольные комплексные постоянные.

\begin{proof} Для любого набора констант $c_j \in \mathbb{C}$ формула (\ref{t3}) определяет решение линейного неоднородного уравнения $\mathcal{L}y = f(t)$ в силу линейности уравнения. Согласно определению общего решения осталось показать, что выбором констант в (\ref{t3}) можно получить любое наперед заданное решение $\mathcal{L}y = f(t)$, то есть для любого решения $\tilde{y}(t)$ неоднородного уравнения $\mathcal{L}y = f(t)$ найдутся константы $\tilde{c}_1, \tilde{c}_2 ,\dots, \tilde{c}_n$ такие, что на отрезке $[a, b]$ будет выполнено равенство
\begin{equation}
    \tilde{y}(t) = y_H(t) + \tilde{c}_1y_1(t) + \tilde{c}_2y_2(t) +\dots + \tilde{c}_ny_n(t).
    \label{t3_2}
\end{equation}

Пусть $\tilde{y}(t)$ --- решение неоднородного уравнения $\mathcal{L}y = f(t)$. Разность $y(t) = \tilde{y}(t)-y_H(t)$ двух решений линейного неоднородного уравнения $\mathcal{L}y = f(t)$ является решением однородного уравнения $\mathcal{L}y = 0$. По теореме об общем решении линейного однородного уравнения найдутся комплексные константы $\tilde{c}_j$ такие, что на рассматриваемом отрезке выполнено равенство $\tilde{y}(t) = \tilde{c}_1y_1(t) + \tilde{c}_2y_2(t) +\dots + \tilde{c}_ny_n(t)$, а вместе с ним и искомое равенство (\ref{t3_2}).
\end{proof}




% -------- source --------
\bigbreak
[\cite[page 65-73]{denisov}]