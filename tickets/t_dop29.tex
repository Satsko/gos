\textbf{\LARGE dop 29. Исследование устойчивости по начальным данным схемы с весами для уpавнения теплопpоводности.}

Обозначим пространство $H_h = {y(x), \, x \in \Omega_h}$ --- пространства функций заданньых на разностной сетке $\Omega_h$, размерность которых зависит от шагов сетки $h$.

\textbf{Определение.}
Пусть заданы линейные операторы $B_1,B_2: H_h \rightarrow H_h$ и функции $\phi_n, y_0 \in H_h.$ \textit{Двуслойной разностной схемой} называется семейство операторно-разностных уравнений
\begin{equation*}
B_1 y_{n+1} + B_2 y_n = \phi_n, \quad n=0,1,\dots ,K-1.
\end{equation*}

\textbf{Определение.}
\textit{Каноническим видом} двуслойной схемы называется ее запись в форме
\begin{equation*}
    B \frac{y_{n+1}-y_n}{\tau} + A y_n = \phi_n, \quad n=0,1,\dots ,K-1. \eqno(1)
\end{equation*}

Далее будем считать, что в канонической форме (1) для всех $h,\tau ,n$ у оператора $B=B_{h,\tau}(t_n)$ существует обратный оператор $B^{-1},$ а также в пространстве $H_h$ заданы нормы $||\cdot||_{1_h}$ и $||\cdot||_{2_h}$.

\textbf{Определение.}
Разностная схема (1) называется \textit{устойчивой}, если существуют константы $M_1,M_2>0$, не зависящие от  $h,\tau ,n$, такие, что при любых правых частях $\phi_{h,\tau}(t_n) \in H_h$ и любых начальных данных $y_0 \in H_h$ для решения уравнения (1) выполняется оценка
\begin{equation*}
    ||y_n||_{1_h} \leq M_1 ||y_0||_{1_h} + M_2  \max_{0 \leq j \leq n-1} |\phi_j||_{2_h}, \quad n=1,2,\dots ,K. \eqno(2)
\end{equation*}

Рассмотрим также однородное уравнение и уравнение с нулевыми начальными данными:
\begin{equation*}
    B \frac{y_{n+1}-y_n}{\tau} + A y_n = 0, \quad n=0,1,\dots ,K-1, \; y_0 \in H_h, \eqno(1')
\end{equation*}
\begin{equation*}
    B \frac{y_{n+1}-y_n}{\tau} + A y_n = \phi_n, \quad n=0,1,\dots ,K-1, \; y_0=0. \eqno(1'')
\end{equation*}

\textbf{Определение.}
Разностная схема (1) называется \textit{устойчивой по начальным данным}, если существует постоянная $M_1>0$, не зависящая от  $h,\tau ,n$, такая, что при любых начальных данных $y_0 \in H_h$ для решения уравнения (1') выполняется оценка
\begin{equation*}
||y_n||_{1_h} \leq M_1 ||y_0||_{1_h}, \quad n=1,2,\dots ,K.
\end{equation*}

\textbf{Определение.}
Разностная схема (1) называется \textit{устойчивой по правой части}, если существует постоянная $M_2>0$, не зависящая от  $h,\tau ,n$, такая, что при любых правых частях $\phi_{h,\tau}(t_n) \in H_h$ для решения уравнения (1'') выполняется оценка
\begin{equation*}
||y_n||_{1_h} \leq M_2  \max_{0 \leq j \leq n-1} |\phi_j||_{2_h}, \quad n=1,2,\dots ,K.
\end{equation*}

\textbf{Определение.}
Разностная схема (1) называется \textit{равномерно устойчивой по начальным данным}, если существует постоянная $\rho>0$ и постоянная $M_1>0$, не зависящая от  $h,\tau ,n$, такие, что при любых начальных данных $y_0 \in H_h$ для решения уравнения (1') выполняется оценка
\begin{equation*}
||y_{n+1}||_{1_h} \leq \rho ||y_n||_{1_h}, \quad n=1,2,\dots ,K,
\end{equation*}
причем $\rho^n \leq M_1.$

\textbf{Теорема.}
Пусть схема (1) равномерно устойчива по начальным данным в норме $||\cdot||_{1_h}$ и $0 \leq n\tau \leq T$. Тогда схема (1) устойчива и по правой части, причем для ее решения выполнена оценка устойчивости
\begin{equation*}
||y_n||_{1_h} \leq M_1 ||y_0||_{1_h} + M_2  \max_{0 \leq j \leq n-1} |\phi_j||_{2_h},
\end{equation*}
в которой $||\phi_j||_{2_h} = ||B_j^{-1} \phi_j||_{1_h}$ и $M_2=M_1 T$.

\textbf{Следствие.} Для устойчивости двуслойной схемы достаточно равномерной устойчивости схемы по начальным данным.

\par
\textbf{Теорема об устойчивости по начальным данным}

Пусть в $H_h$ введены скалярное произведение $(y,v)$, норма $||y||=\sqrt{(y,y)}$ и энергетическая норма $||y||_A=\sqrt{(Ay,y)}$, если $A^*=A>0$.

\textbf{Теорема.}
Пусть в схеме (1) оператор А является самосопряженным, положительно определенным и не зависит от $n$. Тогда при выполнении операторного неравенства $B \geq 0,5\tau A$ схема (1) равномерно устойчива по начальным данным и для решения однородного уравнения (1') справедлива оценка
\begin{equation*}
||y_{n+1}||_A \leq ||y_n||_A, \quad n=0,1,\dots , K-1.
\end{equation*}

\textbf{Уравнение теплопроводности}

Первая краевая задача для одномерного уравнения теплопроводности:
\begin{equation*}
\begin{split}
    \frac{\partial u}{\partial t} = \frac{\partial^2 u}{\partial x^2}, \quad 0<x<l, 0<t \leq T;
    \\
    u(0,t)=\mu_1 (t), u(l,t)=\mu_2 (t); u(x,0) = u_0(x)
\end{split}
\end{equation*}
Используем разностную сетку
\begin{equation*}
\begin{aligned}
    & \Omega_h = \{x_i=ih; \; i=0,1,\dots ,N; \; hN=l\};
    \\
    & \omega_{\tau} = \{t_n = n\tau ; \; n=0,1,\dots , K; \; \tau K = T\}
\end{aligned}
\end{equation*}
Сопоставим исходной задаче разностную схему
\begin{equation*}
    \begin{aligned}
        & \frac{y_i^{n+1}-y_i^n}{\tau}=\sigma y_{\Bar{x}x,i}^{n+1} + (1-\sigma)y_{\Bar{x}x,i}^n,
        \quad i=1,2,\dots ,N-1; \; n=\overline{0,K-1};
        \\
        & y_0^{n+1}=\mu_1(t_{n+1}), \;
        y_N^{n+1}=\mu_2(t_{n+1}); \;
        y_i^0=u_0(x_i) &
    \end{aligned} \eqno(3)
\end{equation*}

Здесь $\sigma$ --- числовой параметр, называемый весом схемы.

Введем пространство $H_h=\{ y(x_i), \, x_i \in \Omega_h, y_0=y_N=0 \}$ со скалярным произведением

$(y,v)=\sum_{i=1}^{N-1} y_i v_i h$ и оператор
$$
A:H_h \rightarrow H_h; \; (Ay)_i = -y_{\Bar{x}x,i}, \; i=1,2,\dots ,N-1; \; y_0=y_N=0
$$
Запишем схему с весами для уравнения теплопроводности в каноническом виде:
\begin{gather*}
    \frac{y_{n+1}-y_n}{\tau} + \sigma A y_{n+1} + (1-\sigma ) A y_n = 0, \text{ т.е. }
    \\
    B \frac{y_{n+1}-y_n}{\tau} + A y_n = \phi_n, \text{ где } B=E+\sigma \tau A, \; \phi_n=0.
\end{gather*}

Поскольку $A^*=A>0$ и не зависит от $n$, для устойчивости схемы достаточно выполнения неравенства
\begin{equation*}
B=E+\sigma \tau A \geq 0,5 \tau A \Leftrightarrow (0,5-\sigma)\tau (Ay,y) \leq ||y||^2 \forall y \in H_h.
\end{equation*}
Учитывая оценку спектра $(Ay,y)<\frac{4}{h^2}||y||^2$, получим достаточное условие устойчивости схемы
\begin{equation*}
    \sigma \geq \frac{1}{2}-\frac{h^2}{4\tau}
\end{equation*}
Это условие является также необходимым условием устойчивости схемы с весами.


% -------- source --------
\bigbreak
[\cite[page 41, 117-122]{chmmf}]
