\textbf{\LARGE osn 9. Ряд Фурье по ортогональной системе функций. Неравенство Бесселя, равенство Парсеваля,  сходимость ряда Фурье.}

Два элемента $f$ и $g$ евклидова пространства называются \textbf{ортогональными}, если скалярное произведение $\langle f, g \rangle = 0$.
Рассмотрим в произвольном бесконечномерном евклидовом пространстве $E$ некоторую последовательность элементов.
\begin{equation}
    \psi_1, \psi_2, \dots, \psi_n, \dots
    \label{asdfkjfnavw}
\end{equation}
Последовательность (\ref{asdfkjfnavw}) называется \textbf{ортонормированной системой}, если входящие в эту последовательность элементы попарно ортогональны и имеют норму, равную единице.

Пусть в произвольном бесконечномерном евклидовом пространстве $E$ задана произвольная ортонормированная система элементов $\{\psi_k\}$. Рассмотрим какой угодно элемент $f$ пространства $E$.

Назовём \textbf{рядом Фурье} элемента $f$ по ортонормированной системе $\{\psi_k\}$ ряд вида
$$\displaystyle\sum_{k=1}^{\infty} f_k\psi_k,$$
в котором через $f_k$ обозначены постоянные числа, называемые \textbf{коэффициентами Фурье} элемента $f$ и определяемые равенствами $f_k =\langle f,\psi_k\rangle, ~ k=1,2,\dots$

$S_n =\displaystyle\sum_{k=1}^{n}f_k\psi_k$ называется $n$-й \textbf{частичной суммой ряда Фурье}.

Рассмотрим наряду с $n$-й частичной суммой произвольную линейную комбинацию первых $n$ элементов ортонормированной системы $\{\psi_k\}$: $\displaystyle\sum_{k=1}^{n}C_k\psi_k$ с какими угодно постоянными числами $C_1, C_2, \dots, C_n$.

Величина $ \| f-g \|$ называется \textbf{отклонением} $f$ по норме данного евклидова пространства.

\textbf{Задача о начальном приближении:} $\displaystyle\min_{\forall\{C_j\}\in\mathbb{R}}  \|f-\displaystyle\sum_{k=1}^{n}C_k\psi_k \|$

Будем минимизировать квадрат нормы:

$ \| f-\displaystyle\sum_{k=1}^{n}C_k\psi_k \|^2 =  
\left\langle f-\displaystyle\sum_{k=1}^{n}C_k\psi_k, ~ f-\displaystyle\sum_{k=1}^{n}C_k\psi_k \right\rangle = 
\langle f, f \rangle - 2\displaystyle\sum_{k=1}^{n}C_k\langle f, \psi_k \rangle + \displaystyle\sum_{k=1}^{n} C_k^2 = 
\| f\|^2 + \displaystyle\sum_{k=1}^{n}(C_k^2-2C_kf_k) = 
\left\{ \pm \displaystyle\sum_{k=1}^{n}f_k^2 \right\} = 
\| f\|^2 - \displaystyle\sum_{k=1}^{n}f_k^2 + \displaystyle\sum_{k=1}^{n}(C_k-f_k)^2$,

минимум достигается при $ C_k = f_k,~k=1,2,\dots,n$. Таким образом, доказана следующая теорема:

\textbf{Теорема.} Среди всевозможных линейных комбинаций элементов ортонормированной системы $\{\psi_k\}$ евклидова пространства наименьшее отклонение от произвольного элемента  $f$ из пространства имеет $n$-я частичная сумма ряда Фурье элемента $f$ по системе $\{\psi_k\}$.

\textbf{Следствие 1.} $\forall$ элемента $f$ данного евклидова пространства, $\forall$ ортонормированной системы $\{\psi_k\}$ при произвольном выборе постоянных $C_k$ и $\forall n$ справедливо неравенство
$$ \| f \|^2-\displaystyle\sum_{k=1}^{n}f_k^2 \leqslant  \| \displaystyle\sum_{k=1}^{n}C_k\psi_k-f \|^2$$

\begin{proof}
$ \| f-\displaystyle\sum_{k=1}^{n}C_k\psi_k \|^2 = 
\| f\|^2 - \displaystyle\sum_{k=1}^{n}f_k^2 + \displaystyle\sum_{k=1}^{n}(C_k-f_k)^2 \geq \| f\|^2 - \displaystyle\sum_{k=1}^{n}f_k^2$
\end{proof}

\textbf{Следствие 2 (тождество Бесселя).} $\forall$ элемента $f$ данного евклидова пространства, $\forall$ ортонормированной системы $\{\psi_k\}$ и $\forall n$ справедливо равенство
$$ \| \displaystyle\sum_{k=1}^{n}f_k\psi_k-f \|^2 = \| f \|^2-\displaystyle\sum_{k=1}^{n}f_k^2$$

\begin{proof}
Подставить $C_k = f_k$ в $ \| f-\displaystyle\sum_{k=1}^{n}C_k\psi_k \|^2 = 
\| f\|^2 - \displaystyle\sum_{k=1}^{n}f_k^2 + \displaystyle\sum_{k=1}^{n}(C_k-f_k)^2$ 
\end{proof}

\textbf{Неравенство Бесселя.} $\forall$ элемента $f$ данного евклидова пространства, $\forall$ ортонормированной системы $\{\psi_k\}$ выполняется неравенство Бесселя: $$ \displaystyle\sum_{k=1}^{n}f_k^2 \leqslant \| f\|^2 $$

\begin{proof}
Из тождества Бесселя: $ \| \displaystyle\sum_{k=1}^{n}f_k\psi_k-f \|^2 \geq 0 \implies \| f \|^2-\displaystyle\sum_{k=1}^{n}f_k^2 \geq 0 \implies \| f \|^2 \geq \displaystyle\sum_{k=1}^{n}f_k^2$
\end{proof}

Ортонормированная система $\{\psi_k\}$ называется \textbf{замкнутой}, если $\forall$ элемента $f$ данного евклидова пространства $E$ и $\forall$ числа $\varepsilon > 0$ найдётся такая линейная комбинация конечного числа элементов $\{\psi_k\}$, отклонение которой от $f$ (по норме пространства $E$) меньше $\varepsilon$.

Другими словами, любой элемент пространства можно с любой степенью точности приблизить по норме этого пространства линейной комбинацией конечного числа первых элементов этой системы.

\textbf{Теорема.} Если ортонормированная система $\{\psi_k\}$ является замкнутой, то $\forall$ элемента $f$ рассматриваемого евклидова пространства неравенство Бесселя переходит в точное равенство
$$\displaystyle\sum_{k=1}^{\infty}f_k^2= \| f  \|^2,$$ называемое \textbf{равенством Парсеваля}.

\begin{proof}
Фиксируем произвольный элемент $f$ рассматриваемого евклидова пространства и произвольное $\varepsilon > 0$. 
Т.к система $f_k$ является замкнутой, то найдётся такой номер $n$ и такие числа $C_1, C_2, \dots, C_n$, что квадрат нормы, стоящий в правой части неравенства из следствия 1, будет меньше $\varepsilon$. В силу следствия 1 это означает, что для произвольного $\varepsilon > 0$ найдётся номер $n$, для которого
$ \| f  \|^2- \displaystyle\sum_{k=1}^{n}f_k^2 < \varepsilon$.

$\forall m > n$ это неравенство будет тем более справедливо, так как при возрастании номера $n$ сумма, стоящая в левой части может только возрасти.
В соединении с неравенством Бесселя это означает, что ряд  сходится к сумме $ \| f \|^2$.
\end{proof}

% -------- source --------
