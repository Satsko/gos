\textbf{\LARGE dop 6. Билинейные и квадpатичные фоpмы. Пpиведение их к каноническому виду. Закон инеpции.}


\begin{definition}
Пусть $V$~--- линейное пространство над $\bbR$. Отображение\\ $\mcA\colon V\times V\to \bbR$ называется \emph{билинейной формой} в пространстве $V$, если
\begin{enumerate}
    \item $\mcA(x+y,z)=\mcA(x,z)+\mcA(y,z)$,
    \item $\mcA(\alpha x,y)=\alpha\mcA(x,y)$,
    \item $\mcA(x,y+z)=\mcA(x,y)+\mcA(x,z)$,
    \item $\mcA(x,\alpha y)=\alpha\mcA(x,y)$,
\end{enumerate}
$\forall x,y,z\in V,~\alpha\in\bbR$.
\end{definition}

\begin{definition}
Билинейная форма называется \emph{симметрической}, если $\mcA(x,y)=\mcA(y,x),\forall x,y\in V$, и \emph{кососимметрической}, если $\mcA(y,x)=-\mcA(x,y),\forall x,y\in V$.
\end{definition}

\begin{example}
В $n$-мерном пространстве $V$ с базисом $e_1,\ldots,e_n$ отображение\\ $\mcA:V\times V\to\bbR$, определенное правилом
\begin{equation}
\label{eq общий вид билин. формы}
    \mcA(x,y)=\sum_{i,j=1}^na_{ij}x_iy_j,
\end{equation}
$\forall x=\sum_{i=1}^nx_ie_i,~y=\sum_{i=1}^ny_ie_i$, где $a_{ij}~(i,j=\overline{1,n})$~--- фиксированные числа, является билинейной формой (в силу линейности координат).
\end{example}

\begin{definition}
Представление билинейной формы в виде (\ref{eq общий вид билин. формы}) называется \emph{общим видом билинейной формы в базисе $e$}.\\
Матрица $A_e=(a_{ij})\in\bbR^{n\times n}:a_{ij}=\mcA(e_i,e_j),~i,j=\overline{1,n}$ называется \emph{матрицей билинейной формы $\mcA(x,y)$ в базисе $e$}.\\
Общий вид билинейной формы может быть записан в компактном виде: если $x_e,~y_e$~--- координатные столбцы векторов $x$ и $y$ в базисе $e$, то
\begin{equation}
\label{eq комп. вид билин. формы}
    \mcA(x,y)=x_e^TA_ey_e,~\mcA(x,y)=y_e^TA_e^Tx_e.
\end{equation}
Первое из равенств проверяется непосредственно, второе можно получить транспонированием обеих частей первого.
\end{definition}

\begin{definition}
Пусть $\mcA(x,y)$~--- симметрическая билинейная форма в пространстве $V$ над полем $\bbP$. \emph{Квадратичной формой} называется отображение $\mcA\colon V\to\bbP$, которое $\forall x\in V\mapsto \mcA(x,x)$, то есть сужение симметрической билинейной формы на диагональ декартова квадрата $V\times V$.\\
\textsf{Обозначение:}~$\mcA(x,x)$ или $\mcA(x)$
\end{definition}

\begin{definition}
Билинейная форма $\mcA(x,y)$ при этом называется \emph{полярной билинейной формой} к квадратичной форме $\mcA(x,x)$.
\end{definition}

\begin{Commentwhite}
Из свойств билинейных форм следует:
\begin{enumerate}
    \item В базисе $e$ квадратичная форма $\mcA(x,x)$ с матрицей $A_e=(a_{ij})$ может быть записана в виде: $\forall x=\sum_{i=1}^nx_ie_i$
    \begin{equation}
    \label{eq общий вид кв. формы}
        \mcA(x,x)=\sum_{i,j=1}^na_{ij}x_ix_j,~a_{ij}=a_{ji},
    \end{equation}
    \item $rg\mcA(x,x)=rg\mcA(x,y)$.
\end{enumerate}
\end{Commentwhite}

\begin{definition}
Базис $e=(e_1,\ldots,e_n)$ называется \emph{каноническим базисом квадратичной формы} $\mcA(x,x)$, если её матрица в этом базисе диагональна, то есть $A_e=diag(\lam_1,\ldots,\lam_n)$.
\end{definition}

\begin{definition}
В каноническом базисе квадратичная форма $\mcA(x,x)$ имеет вид $\mcA(x,x)=\lam_1x_1^2+\ldots+\lam_nx_n^2$, который называется \emph{каноническим видом} квадратичной формы или \emph{суммой квадратов}. Числа $\lam_1,\ldots,\lam_n$ называются её \emph{каноническими} коэффициентами.
\end{definition}

\begin{Commentwhite}
Очевидно, число ненулевых квадратов совпадает с рангом $\mcA(x,x)$. Итак, если $e$~--- канонический базис и $r=rg\mcA(x,x)$, то
\begin{equation}
\label{eq канон. вид квадр. формы}
    \mcA(x,x)=\lam_1x_1^2+\ldots+\lam_rx_r^2,~\forall x=\sum_{i=1}^nx_ie_i.
\end{equation}
\end{Commentwhite}

\begin{theorem}
\label{th метод Лагранжа}
(Метод Лагранжа приведения к каноническому виду)\\
Для любой квадратичной формы существует канонический базис.
\end{theorem}
\begin{proofocre}
~\\
Пусть $e=(e_1,\ldots,e_n)$~--- базис $V$. Квадратичная форма $\mcA(x,x)$ с матрицей $A_e=\\=(a_{ij})$ имеет в этом базисе вид (\ref{eq общий вид кв. формы}). Обозначим $g(x_1,\ldots,x_n)=\sum_{i,j=1}^na_{ij}x_ix_j$.\\
Покажем, что переходом к другому базису квадратичная форма $\mcA(x,x)$ приводится к каноническому виду.\\
Пусть $A_e\neq O$ (если это так, то $e$~--- искомый базис). Обозначим через $\Delta_k$ угловые миноры $k$-го порядка матрицы $A_e$, то есть $\Delta_k=M_{1,\ldots,k}^{1,\ldots,k},~k=\overline{1,n}$, и положим $\Delta_0=1$.
\begin{enumerate}
    \item Рассмотрим сначала случай, когда $\Delta_k\neq 0,~k=\overline{1,n}$. В этом случае процесс состоит из $n-1$ однотипных шагов.\\
    \emph{Первый шаг} основан на том, что $\Delta_1\neq 0$, то есть $a_{11}\neq 0$. Сгруппируем все члены квадратичной формы $g(x_1,\ldots,x_n)$, содержащие $x_1$, и выделим из них полный квадрат:\\
    $g(x_1,\ldots,x_n)=a_{11}x_1^2+2\sum_{k=2}^na_{1k}x_1x_k+\sum_{i,k=2}^na_{ik}x_ix_k=\\=a_{11}\left(x_1+\sum_{k=2}^n\frac{a_{1k}}{a_{11}}x_k \right)^2-a_{11}\left(\sum_{k=2}^n\frac{a_{1k}}{a_{11}}x_k\right)^2+\sum_{i,k=2}^na_{ik}x_ix_k.$\\
    Перейдем к новым координатам:
    $x'_1=x_1+\sum_{k=2}^n\frac{a_{1k}}{a_{11}}x_k$ и $x'_j=x_j$ при $j\neq 1$,\\
    очевидно выполнив при этом невырожденное преобразование координат с матрицей
    \begin{equation}
        Q_1=
        \begin{bmatrix}
            1 & -\frac{a_{12}}{a_{11}} & \cdots & -\frac{a_{1n}}{a_{11}}\\
            0 & 1 & \cdots & 0\\
            \vdots & \vdots & \ddots & \vdots\\
            0 & 0 & \cdots & 1
        \end{bmatrix}.
    \end{equation}
    Тогда в новых координатах квадратичная форма имеет вид $\mcA(x,x)=\\=a_{11}x_1^2+h(x'_2,\ldots,x'_n)$, где $h(x'_2,\ldots,x'_n)$~--- квадратичная форма от переменных $x'_2,\ldots,x'_n$. При этом матрица $A_1=Q_1^TA_eQ_1$ квадратичной формы $\mcA(x,x)$ в новом базисе будет иметь вид
    $$A_1=
    \begin{bmatrix}
        a_{11} & 0 & \cdots & 0\\
        0 & a'_{22} & \cdots & a'_{2n}\\
        \vdots & \vdots & \ddots & \vdots\\
        0 & a'_{n2} & \cdots & a'_{nn}
    \end{bmatrix}.$$
    Каждая строка (столбец) матрицы $A_1$, начиная со второй, получена из соответствующей строки (столбца) матрицы $A_e$ вычитанием из неё первой строки (столбца) матрицы $A_e$, умноженной на некоторое число, поэтому угловые миноры матрицы $A_1$ совпадают с $\Delta_1,\ldots,\Delta_n \Rightarrow \Delta_1=a_{11},\Delta_2=\\=a_{11}a'_{22}$ и
    \begin{equation}
    \label{eq a22}
        a'_{22}=\frac{\Delta_2}{\Delta_1}\neq 0.
    \end{equation}
    \emph{Второй шаг} основан на том, что $\Delta_2\neq 0$, то есть $a'_{22}\neq 0$, и состоит в применении действий первого шага к квадратичной форме $h(x'_2,\ldots,x'_n)$: выделяется полный квадрат среди всех членов, содержащих $x'_2$, выполняется невырожденное преобразование координат
    $$x''_2=x'_2+\sum_{k=3}^n\frac{a'_{2k}}{a'_{22}}x'_k~\text{и}~x''_j=x'_j~\text{при}~j\neq 2$$
    и квадратичная форма $\mcA(x,x)$ приводится к виду $\mcA(x,x)=a_{11}{x_1''}^2+\\+a'_{22}{x_2''}^2+v(x''_3,\ldots,x''_n)$, где $v(x''_3,\ldots,x''_n)$~--- квадратичная форма от переменных $x''_3,\ldots,x''_n$, а ее матрица --- к виду
    $$A_2=
    \begin{bmatrix}
        a_{11} & 0 & 0 & \cdots & 0\\
        0 & a'_{22} & 0 & \cdots & 0\\
        0 & 0 & a''_{33} & \cdots & a''_{3n}\\
        \vdots & \vdots & \vdots & \ddots & \vdots\\
        0 & 0 & a''_{n3} & \cdots & a''_{nn}
    \end{bmatrix}.$$
    По-прежнему угловые миноры матрицы $A_2$ совпадают с $\Delta_1,\ldots,\Delta_n$, поэтому
    \begin{equation}
    \label{eq a33}
        a''_{33}=\frac{\Delta_3}{\Delta_2}\neq 0.
    \end{equation}
    Повторяя этот процесс, через $(n-1)$ шагов мы придем к базису, в котором матрица квадратичной формы $\mcA(x,x)$ имеет диагональную форму: $A_{n-1}=\\=diag(\lam_1,\ldots,\lam_n)$, где с учетом (\ref{eq a22}) и (\ref{eq a33}) и обозначения $\Delta_0=1$
    \begin{equation}
    \label{eq лямбды у квадр.}
        \lam_i=\frac{\Delta_i}{\Delta_{i-1}},~i=\overline{1,n}.
    \end{equation}
    Каждый $i$-ый шаг процесса начинается с условия, что $\Delta_i\neq 0$, так как только при выполнении этого условия можно провести текущий шаг. Метод, описанный выше, называется \emph{методом Лагранжа}.
    \item Пусть теперь сред угловых миноров $\Delta_1,\ldots,\Delta_{n-1}$ могут встретиться нулевые. Модифицируем метод Лагранжа.\\
    Опишем $i$-ый шаг. Пусть после $(i-1)$-го шага матрица квадратичной формы $\mcA(x,x)$ имеет вид
    $$A_{i-1}=
    \begin{bmatrix}
        a_{11} & & & O\\
        & \ddots & &\\
        & & a_{i-1,i-1} &\\
        O & & & C
    \end{bmatrix},~\text{где}~C=
    \begin{bmatrix}
        a_{ii} & \cdots & a_{in}\\
        \vdots & \ddots & \vdots\\
        a_{ni} & \cdots & a_{nn}
    \end{bmatrix}$$
    (штрихи опущены для упрощения записи). Будем считать, что $C\neq O$ (если это так, то канонический базис уже построен).
    \begin{enumerate}
        \item Если $a_{ii}\neq 0$, то выполним $i$-ый шаг метода Лагранжа.
        \item Пусть $a_{ii}=0$.\\
        а) Если среди диагональных элементов матрицы $C~\exists~a_{jj}\neq 0,~j>i$, то перенумеруем переменные (то есть векторы базиса): $x'_i=x_j,~x'_j=\\=x_i,~x'_k=x_k$ при $k\neq i,j$. Тогда в матрице $A_{i-1}$ поменяются местами строки (столбцы) с номерами $i$ и $j$, поэтому в позиции $(i,i)$ окажется ненулевой элемент $a'_{ii}=a_{jj}$, с помощью которого выполним $i$-ый шаг метода Лагранжа.\\
        б) Пусть все диагональные элементы матрицы $C$ равны нулю, тогда в ней $\exists~a_{kj}\neq 0,~k,j\ge i,~k\neq j$. Это означает, что в квадратичной форме от переменных $x_i,\ldots,x_n$ отсутствуют квадраты $x_i^2,\ldots,x_n^2$, но содержится член вида $2a_{kj}x_kx_j$. Перейдем к новым координатам, положив $x_k=x'_k+x'_j,~x_j=x'_k-x'_j$ и $x'_s=x_s$ при $s\neq k,j$. Тогда квадратичная форма будет иметь квадраты ${x_k'}^2,{x_j'}^2$ и мы окажемся в ситуации (a), при этом все преобразования координат были невырожденными.
    \end{enumerate}
\end{enumerate}
\end{proofocre}

\begin{definition}
Пусть квадратичная форма $\mcA(x,x)$ приведена к каноническому виду (\ref{eq канон. вид квадр. формы}). Число $\pi$ положительных квадратов в (\ref{eq канон. вид квадр. формы}) и число $\nu=r-\pi$ называются \emph{положительным и отрицательным индексами инерции} квадратичной формы $\mcA$, а их разность $\sigma=\pi-\nu$ называется \emph{сигнатурой} $\mcA(x,x)$.
\end{definition}

\begin{theorem}
\label{th закон инерции}
(Закон инерции)\\
Положительный и отрицательный индексы инерции вещественной квадратичной формы не зависят от выбора канонического базиса.
\end{theorem}
\begin{proofocre}
~\\
Пусть $e$ и $f$~--- канонические базисы для $\mcA(x,x)$ ранга $r$ и для\\ $x=\sum_{i=1}^nx_ie_i=\sum_{i=1}^ny_if_i$
\begin{equation}
\label{eq 43.1}
\begin{gathered}
    \mcA(x,x)=a_1x_1^2+\ldots+a_px_p^2-a_{p+1}x_{p+1}^2-\ldots-a_rx_r^2,\\
    \mcA(x,x)=b_1y_1^2+\ldots+b_qy_q^2-b_{q+1}y_{q+1}^2-\ldots-b_ry_r^2,
\end{gathered}
\end{equation}
где $a_i>0,~b_i>0,~i=\overline{1,r}$.\\
Докажем, что $p\le q$. От противного: пусть $p>q$.\\
Рассмотрим подпространства $L_1=\mcL(e_1,\ldots,e_p)$ и $L_2=\mcL(f_{q+1},\ldots,f_n)$. Согласно $\dim (L_1+L_2) = \dim L_1 + \dim L_2 - \dim(L_1 \cap L_2)$, $dim(L_1\cap L_2)=p+(n-q)-dim(L_1+L_2);~dim(L_1+L_2)\le n,\\p>q \Rightarrow dim(L_1+L_2)>0 \Rightarrow\exists~x_0\neq\theta,~x_0\in L_1\cap L_2$.\\
Пусть $x_0=\alpha_1e_1+\ldots+\alpha_pe_p=\beta_{q+1}f_{q+1}+\ldots+\beta_nf_n$. Тогда, согласно (\ref{eq 43.1})
\begin{equation}
    \mcA(x_0,x_0)=a_1\alpha_1^2+\ldots+a_p\alpha_p^2=-b_{q+1}\beta_{q+1}^2-\ldots-b_r\beta_r^2.
\end{equation}
Так как $x_0\neq\theta$, то $a_1\alpha_1^2+\ldots+a_p\alpha_p^2>0,~-b_{q+1}\beta_{q+1}^2-\ldots-b_r\beta_r^2<0 \Rightarrow$ (!) $\Rightarrow p\le q$.\\
Аналогично доказывается, что $p\ge q$. Значит, $p=q$.
\end{proofocre}


% -------- source --------
\bigbreak
[\cite{kim}]
