\textbf{\LARGE osn 26. Теоремы существования и единственности решения задачи Коши для обыкновенного дифференциального уравнения первого порядка, разрешенного относительно производной.}

Пусть функция $f(t,y)$ определена и непрерывна в прямоугольнике 
П$ = \{(t, y) : |t - t_0| \leqslant T, |y - y_0| \leqslant A\}$.

Рассмотрим на отрезке $[t_0 - T ; t_0 + T ]$ дифференциальное уравнение с условием:
\begin{equation}
    y'(t) = f(t, y(t))
    \label{eq3}
\end{equation}
\begin{equation}
    y(t_0) = y_0
    \label{eq4}
\end{equation}

Требуется определить функцию $y(t)$, удовлетворяющую уравнению (\ref{eq3}) и условию (\ref{eq4}).

Эта задача называется \textbf{задачей с начальным условием или задачей Коши}. Рассмотрим отрезок $[t_1, t_2]$ такой, что $t_0 - T \leqslant t_1 < t_2 \leqslant t_0 + T,~t_0 \in [t_1, t_2]$.

\textbf{Опр.} Функция $y(t)$ называется \textbf{решением задачи Коши} (\ref{eq3}), (\ref{eq4}) на отрезке $[t_1, t_2]$, если $y(t) \in C^1[t_1, t_2],~|y(t) - y_0| \leqslant A$ для $t \in [t_1, t_2],~y(t)$ удовлетворяет уравнению (\ref{eq3}) для $t \in [t_1, t_2]$ и (\ref{eq4}).

Рассмотрим на отрезке $[t_0 - T,t_0 + T]$ уравнение относительно неизвестной функции $y(t)$:
\begin{equation}
    y(t) = y_0 + \int\limits_{t_0}^{t}f(\tau, y(\tau))d\tau
    \label{eq5}
\end{equation}

\textbf{Лемма 1.} Функция $y(t)$ является решением задачи Коши (\ref{eq3}), (\ref{eq4}) на отрезке $[t_1, t_2] \iff$ когда $y(t) \in C[t_1, t_2],~|y(t) - y_0| \leqslant A$ для $t \in [t_1, t_2]$ и $y(t)$ удовлетворяет уравнению (\ref{eq5}) для $t \in [t_1, t_2]$.

\begin{proof} ($\implies$) Пусть функция $\overline{y}(t)$ является решением задачи с начальным условием (\ref{eq3}), (\ref{eq4}) на отрезке $[t_1, t_2]$. Из определения решения следует, что $\overline{y}(t) \in C[t_1,t_2],~|\overline{y}(t)-y_0| \leqslant A$ для $t \in [t_1,t_2]$. Покажем, что $\overline{y}(t)$ удовлетворяет уравнению (\ref{eq5}) для $t \in [t_1, t_2]$. Интегрируя (\ref{eq3}) от $t_0$ до $t$ получим:
$$ \int\limits_{t_0}^{t}\overline{y}'(\tau)d\tau = \int\limits_{t_0}^{t}f(\tau, \overline{y}(\tau))d\tau,~t\in[t_1,t_2]$$
Учитывая условие (\ref{eq4}), имеем:
$$ \overline{y}(t) = y_0 + \int\limits_{t_0}^{t}f(\tau, \overline{y}(\tau))d\tau,~t\in[t_1,t_2]$$
Следовательно, функция $\overline{y}(t)$ удовлетворяет интегральному уравнению (\ref{eq5}) при $t \in [t_1, t_2]$.

($\impliedby$) Пусть функция $\overline{y}(t)$ такова, что $\overline{y}(t) \in C[t_1, t_2],~|y(t) - y_0| \leqslant A$ для $t \in [t_1, t_2]$ и $\overline{y}(t)$ удовлетворяет уравнению (\ref{eq5}) для $t \in [t_1, t_2]$, то есть:
\begin{equation}
    \overline{y}(t) = y_0 + \int\limits_{t_0}^{t}f(\tau, \overline{y}(\tau))d\tau,~t\in[t_1,t_2]
    \label{eq6}
\end{equation}
Покажем, что $y(t)$ является решением задачи с начальным условием (\ref{eq3}), (\ref{eq4}). 

Положив в (\ref{eq6}) $t = t_0$, получим, что $\overline{y}(0) = y_0$. Следовательно условие (\ref{eq4}) выполнено. Так как функция $\overline{y}(t)$ непрерывна на $[t_1,t_2]$, то правая часть равенства (\ref{eq6}) непрерывно дифференцируема на $[t_1, t_2]$ как интеграл с переменным верхним пределом $t$ от непрерывной функции $f(\tau,\overline{y}(\tau)) \in C[t_1,t_2]$. Следовательно, $\overline{y}(t)$ непрерывно дифференцируема на $[t_1, t_2]$. Дифференцируя (\ref{eq6}), получим, что $\overline{y}(t)$ удовлетворяет (\ref{eq3}).
\end{proof}

\textbf{Опр.} Функция $f(t,y)$, заданная в прямоугольнике П, \textit{удовлетворяет в П условию Липшица} по $y$, если $|f(t,y_1)-f(t,y_2)| \leqslant L|y_1 -y_2|,~\forall(t,y_1),(t,y_2) \in$ П, где $L$ --- положительная постоянная.

\textbf{Лемма Гронуолла-Беллмана.} Пусть функция $z(t) \in C[a,b]$ и такова, что $0 \leqslant z(t) \leqslant c+d \Big|\int\limits_{t_0}^{t}z(\tau)d\tau\Big|,~t\in[a, b]$, где постоянная $c$ неотрицательна, постоянная $d$ положительна, а $t_0$ --- произвольное фиксированное число на отрезке $[a, b]$. Тогда $z(t) \leqslant c e^{d|t-t_0|},~t \in [a, b]$.

\textbf{Теорема (единственности).} Пусть функция $f(t,y)$ непрерывна в П и удовлетворяет в П условию Липшица по $y$.
Если $y_1(t),~y_2(t)$ --- решения задачи Коши (\ref{eq3}), (\ref{eq4}) на отрезке $[t_1, t_2]$, то $y_1(t) = y_2(t)$ для $t \in [t_1, t_2]$.

\begin{proof} Так как $y_1(t)$ и $y_2(t)$ --- решения задачи Коши (\ref{eq3}), (\ref{eq4}), то из Леммы 1 следует, что они являются решениями интегрального уравнения (\ref{eq5}).

То есть:
$$y_1(t) = y_0 + \int\limits_{t_0}^{t}f(\tau, y_1(\tau))d\tau,~t\in[t_1,t_2],$$
$$y_2(t) = y_0 + \int\limits_{t_0}^{t}f(\tau, y_2(\tau))d\tau,~t\in[t_1,t_2].$$
Вычитая второе уравнение из первого и оценивая разность по модулю и используя условие Липшица, получаем:
$|y_1(t) - y_2(t)| = \Big| \int\limits_{t_0}^{t}f(\tau, y_1(\tau))d\tau - \int\limits_{t_0}^{t}f(\tau, y_2(\tau))d\tau \Big| \leq \Big| \int\limits_{t_0}^{t}|f(\tau, y_1(\tau)) - f(\tau, y_2(\tau))|d\tau \Big| \leq L\Big| \int\limits_{t_0}^{t}|y_1(\tau) - y_2(\tau)|d\tau \Big|$

Обозначив $z(t) = |y_1(t) - y_2(t)|$, перепишем последнее неравенство следующим образом:

$0 \leqslant z(t) \leqslant L\Big|\int\limits_{t_0}^{t}z(\tau)d\tau \Big|,~t \in [t_1,t_2].$

Применяя лемму Гронуолла-Беллмана с $c = 0$ и $d = L$, имеем
$z(t) = 0,~t \in [t_1, t_2]$. Следовательно, $y_1(t) = y_2(t),~t \in [t_1, t_2].$
\end{proof} 

\textbf{Теорема (существования) ((локальная)).}
Пусть функция $f(t,y)$ непрерыв на в П, удовлетворяет в П условию Липшица по $y$ и $|f(t,y)| \leq M, (t,y) \in$ П. Тогда на отрезке $[t_o - h, t_0 + h]$, где $h = min\{T, \frac{A}{M}\}$, существует функция $y(t)$ такая, что $y(t) \in C^1 [t_0-h, t_0 + h], |y(t) - y_0| \leq A, t \in [t_0 - h, t_0 + h]$,\newline
$y'(t) = f(t,y(t)), t \in [t_0 - h, t_0 + h]$
$y(t_0) = y_0$\newline

\textit{Следует отметить, что мы можем доказать теорему существования не на всем исходном отрезке $[t0 - T , t0 + T ]$, а на некотором, вообще говоря, меньшем. Поэтому эта теорема часто называется} \textbf{локальной} \textit{теоремой существования решения задачи Коши.}



% -------- source --------
\bigbreak
[\cite[page 25-30]{denisov}]