\documentclass{article}
\usepackage{graphicx} % Required for inserting images
\usepackage{amsthm}
\usepackage[T2A]{fontenc}
\usepackage[utf8]{inputenc}
\usepackage[russian]{babel}
\usepackage{csquotes}
\usepackage{vmargin}
\usepackage{amsmath}
\usepackage{amsfonts}
\usepackage{amssymb}
\usepackage{listings}
\usepackage{enumitem}
\usepackage{accents}
\newtheorem{definition}{Определение}
\newtheorem{theorem}{Теорема}[]
\newtheorem{sle}{Следствие}


\begin{document}

\textbf{\LARGE dop.5 Ряд Лоpана. Классификация изолиpованных особых точек.}
\section{Ряд Лорана}
\begin{definition}
\textbf{Рядом Лорана} называется функциональный ряд вида: $\sum_{n=-\infty}^{n=\infty}c_n(z-z_0)^n\text{(1)}=\sum_{n=0}^{n=\infty}c_n(z-z_0)^n + \sum_{n=-\infty}^{n=-1}c_{n}(z-z_0)^n = \sum_{n=0}^{n=\infty}c_n(z-z_0)^n(\text{ряд.1}) +  \sum_{n=1}^{n=\infty}\frac{c_{-n}}{(z-z_0)^n}(\text{ряд.2})$, где $z$ переменная $\in \mathbb{C}\backslash\{z_0\}$, а $c_n$ коэффициенты $\in \mathbb{C}$.
\end{definition}

\noindent Говорят, что ряд (1) сходится в т. $z$, если в ней сходятся ряд.1 и ряд.2.

\begin{itemize}
    \item Ряд.1 -- обычный степенной ряд. Если его радиус сходимости $R_1=0$, то он сходится лишь в т.$z_0$, а ряд (1) не сходится нигде. Если $R_1>0$, то в круге $|z-z_0|<R_1$ ряд.1 сходится абсолютно к некоторой функции $f_1(z)$.
    
    \item Ряд.2 не является степенным рядом, но приводится к нему заменой $\rho = \frac{1}{z-z_0}$. Если радиус сходимости ряда $\sum_{n=1}^{n=\infty}c_{-n}\rho^n$(2) равен 0, то и ряд (1) и ряд.2 не сходятся. Если радиус сходимости (2) $R^{-1}_2 > 0$, то ряд (2) сходится абсолютно в круге $|\rho|<R^{-1}_2$ $\Rightarrow$ ряд.2 сходится абсолютно в области $|z-z_0|>R_2$  к некоторой функции $f_2(z)$. Если $R_1<R_2$, то области сход. рядов не пересекаются и ряд Лорана не сходится нигде. Если $R_1=R_2=R$, то общие точки сходимости могут лежать лишь на $|z-z_0|=R$ и их наличие требует отдельного исследования. Если $R_1>R_2$, то оба ряда абсолютно сходятся в кольце $D: R_2<|z-z_0|<R_1$, ряд (1) абсолютно сходится там же к функции $f(z) = f_1(z)+f_2(z)$.
\end{itemize}

\begin{definition}
Ряд.1 называют \textbf{правильной} частью ряда Лорана.
\end{definition}

\begin{definition}
Ряд.2 называют \textbf{главной} частью ряда Лорана.
\end{definition}

\textbf{Замечание:} Пусть ряд (1) абс.cход в кольце $D$ к функции $f(z)$. Покажем, что коэффициенты этого ряда однозначно определяются его суммой $f(z)$. Рассмотрим ряд (1) в точках окружности $\phi: |z-z_0|=\rho$, где $R_2<\rho<R_1$. На этой окружности как на компакте, ряд сх-ся равномерно. Равномерная сходимость сохраняется при умножении каждого члена ряда на функцию ограниченную на $\phi$. Фикс $k$ и рассмотрим функцию $\frac{1}{2\pi i(z-z_0)^{k+1}} \Rightarrow \frac{f(z)}{2\pi i(z-z_0)^{k+1}} = \sum_{n=-\infty}^{n=\infty}\frac{1}{2\pi i}c_n(z-z_0)^{n-k-1} \Longleftrightarrow \frac{1}{2\pi i} \oint_\phi \frac{f(z)}{(z-z_0)^{k+1}} \,dz = \sum_{n=-\infty}^{n=\infty}\frac{1}{2\pi i}c_n\oint_\phi(z-z_0)^{n-k-1} \,dz$. Интеграл в правой части $\neq0$ только при $n-k-1=-1 \Longleftrightarrow n=k$(в лекциях Домриной считался) при этом он равен $2\pi i \Rightarrow c_k = \frac{1}{2\pi i} \oint_\phi \frac{f(z)}{(z-z_0)^{k+1}} \,dz$ определены однозначно.

\begin{theorem}
    Функция $f(z)\in A(D), D: R_2<|z-z_0|<R_1$, может быть представлена рядом Лорана по степеням $(z-z_0)$ причем это представление единственно.
\end{theorem}
\begin{proof}
$\Longrightarrow$: доказано в замечании.\\
$\Longleftarrow$ : Фикс произвольную точку $z \in D$, построим вспомогательное кольцо $D'$ c тем же центром в $z_0$, $D' \subset D$ и $z \subset int(D')$. Пусть $\text{Г}_1^{'}: |\rho - z_0|=R_1'$ и $\text{Г}_2^{'}|\rho - z_0|=R_2'$ -- внутрення и внешняя границы кольца $D'$, тогда $f(z) = \frac{1}{2\pi i} \oint_{ \text{Г}_1^{'}} \frac{f(\rho)}{\rho - z} \,d\rho - \frac{1}{2\pi i} \oint_{ \text{Г}_2^{'}} \frac{f(\rho)}{\rho - z} \,d\rho(*)$. Т.к $|\frac{z-z_0}{\rho-z_0}|<1$ для $\forall$ точек $\rho \in \text{Г}_1^{'}$, то подынтегральную дробь $\frac{1}{\rho-z}$ можно заменить $\infty$ геом.прогрессией $\frac{1}{\rho-z} = \frac{1}{\rho-z_0+z_0-z}=\frac{1}{\rho-z_0}\cdot\frac{1}{1-\frac{z-z_0}{\rho-z_0}} = \frac{1}{\rho-z_0}\sum_{n=0}^{\infty}\frac{(z-z_0)^n}{(\rho-z_0)^n} = \sum_{n=0}^{\infty}\frac{(z-z_0)^n}{(\rho-z_0)^{n+1}} \Longleftrightarrow \frac{f(\rho)}{\rho-z_0}=\sum_{n=0}^{\infty}\frac{f(\rho)(z-z_0)^n}{(\rho-z_0)^{n+1}}(**)$ Ряд в правой части сходится равномерно на $\text{Г}_1^{'}$ т.к мажорируется $\max_{\rho \in \text{Г}_1^{'}}|f(\rho)|\sum_{n=0}^{\infty}\frac{(z-z_0)^n}{(\rho-z_0)^{n+1}} \Rightarrow$ можно почленно интегрировать (**) по окружности $\text{Г}_1^{'}$: $\oint_{ \text{Г}_1^{'}}\frac{f(\rho)}{\rho-z_0} \,d\rho =\sum_{n=0}^{\infty} \oint_{ \text{Г}_1^{'}} \frac{f(\rho)(z-z_0)^n}{(\rho-z_0)^{n+1}} \,d\rho \Longleftrightarrow \{ c_n = \frac{1}{2\pi i} \oint_{ \text{Г}_1^{'}} \frac{f(\rho)}{(\rho-z_0)^{n+1}} \,d\rho\ \text{, }n=0..\infty(***)\}\Longleftrightarrow$  $\frac{1}{2\pi i}\oint_{ \text{Г}_1^{'}}\frac{f(\rho)}{\rho-z_0} \,d\rho = \sum_{n=0}^{\infty}c_n(z-z_0)^n(****)$
\bigbreak
\noindentРассмотрим второй интеграл в (*). Для $\forall$ точки $\rho \in  \text{Г}_2^{'}$ выполнено $\mu=\frac{|\rho-z_0|}{|z-z_0|}<1 \Rightarrow -\frac{1}{\rho-z} = \frac{1}{z-z_0-(\rho-z_0)} = \frac{1}{z-z_0}\frac{1}{1-\frac{\rho-z_0}{z-z_0}}=\sum_{n=0}^{\infty}\frac{(\rho-z_0)^n}{(z-z_0)^{n+1}}=\sum_{n=1}^{\infty}\frac{(\rho-z_0)^{n-1}}{(z-z_0)^{n}}=
\sum_{n=1}^{\infty}\frac{(z-z_0)^{-n} }{(\rho-z_0)^{-n+1}}(+)$. Получается р-но.сх-ся ряд на $\text{Г}_2^{'}$ т.к мажорируется числовой прогрессией со знаменателем $\mu$. Равномерная сходимость (+) сохранится и после умножения каждого члена на ограниченную в $\text{Г}_2^{'}$ ф-цию $\frac{f(\rho)}{2\pi i}$. Интегрируя почленно $-\frac{f(\rho)}{2\pi i(\rho-z)}=\sum_{n=1}^{\infty}\frac{f(\rho)(z-z_0)^{-n} }{(\rho-z_0)^{-n+1}}$ по окружности $\text{Г}_2^{'}$ и полагая $c_{-n} = \frac{1}{2\pi i} \oint_{ \text{Г}_1^{'}} \frac{f(\rho)}{(\rho-z_0)^{-n+1}} \,d\rho \text{, }n=1..\infty (++)$. Имеем $\frac{1}{2\pi i} \oint_{ \text{Г}_2^{'}} \frac{f(\rho)}{\rho-z} \,d\rho = \sum_{n=1}^{\infty} c^{-n}(z-z_0)^{-n}(+++)$. Заменяя оба интеграла в (*) на их разложения (****) и (+++) приходим к ряду Лорана.
\end{proof}

\section{Классификация изолированных особых точек}
Пусть $D:0<|z-z_0|<R$-проколотая окрестность точки $z_0\neq\infty$ и $f(z) \in A(D)$. Точка $z_0$ для ф-ции $f(z)$ является \textbf{изолированной особой точкой}. $D$ можно рассматривать как кольцо с центром в т.$z_0$ и внутренним радиусом 0. По теореме Лорана $f(z)$ может быть разложена в $D$ в ряд Лорана по степеням $z-z_0$: $\sum_{n=-\infty}^{n=\infty}c_n(z-z_0)^n\text{(1)}=\sum_{n=0}^{n=\infty}c_n(z-z_0)^n +\sum_{n=1}^{n=\infty}c_{-n}(z-z_0)^{-n}$, $z \in D(1)$.
Для этого ряда имеются 4 возможности:
\begin{enumerate}
    \item Точка $z_0$ – \textbf{устранимая особая точка} $f(z)$, если главная часть ряда Лорана (1) равна нулю.
    \item Точка $z_0$ – \textbf{полюс} $f(z)$, если главная часть ряда Лорана (1) содержит конечное число членов.    
    \item Точка $z_0$ – \textbf{полюс порядка} $k (k \in N)$ функции $f(z)$, если $k$
– максимальная по модулю степень у ненулевого члена главной части лорановского разложения в проколотой окрестности точки $z_0$. А именно, ряд (1) имеет коэффициент $c_{-k}\neq 0$,
в то время как $c_{-n}$ = 0 $\forall n > k$.
    \item Точка $z_0$ – \textbf{существенно особая точка} $f(z)$, если
главная часть ряда (1) содержит бесконечное число членов.
\end{enumerate}
\begin{theorem}
   Следующие 3 утверждения эквивалентны: a) $z_0$ - устранимая особая точка ф-ции $f(z)$, б) $\exists$ конечный $\displaystyle\lim_{z\rightarrow z_0}f(z)$, в) $f(z)$ ограничена в некоторой окрестности точки $z$.
\end{theorem}
\begin{proof}
    a)$\rightarrow$б): По условию $f(z)=\sum_{n=0}^{\infty}c_n(z-z_0)^n$, $z \in D$. Сумма $g(z)$ стоящего справа ряда непрерывна в т.$z_0$ и ее значение в этой точке равно свободному члену $c_0$ ряда, т.к вне $z_0$ функции $f(z)$ и $g(z)$ совпадают, то $\exists \displaystyle\lim_{z\rightarrow z_0}f(z)=c_0$.
    
    \noindentб)$\rightarrow$в) Функция имеющая конечный $lim$ в точке $z_0$ ограничена в некоторой окрестности этой точки.

    \noindentв)$\rightarrow$а) По условию в некоторой окрестности $U$ точки $z_0$ выполняется соотношение $|f(z)|\leq M \forall z \in U$. Пусть $\gamma:|z-z_0|=\rho$ - окружность принадлежащая этой окрестности. Как $\Rightarrow$ из доказательства т.Лорана коэффициенты ряда (1) представимы в виде: $c_n = \frac{1}{2\pi i} \oint_{\gamma} \frac{f(z)}{(z-z_0)^{n+1}} \,dz \Rightarrow |c_n|\leq M\rho^{-1}$. Для отрицательных $n$ правая часть этой оценки стремится к 0 при $\rho \rightarrow 0$. Таким образом в ряде (1) все коэффициенты $c_n$ с отрицательными индексами $=0\Rightarrow z_0$ устранимая особая точка $f(z)$. 
\end{proof}

\begin{theorem}
   Изолир.особая точка $z_0$ ф-ции $f(z)$ является ее полюсом $\Longleftrightarrow \displaystyle\lim_{z\rightarrow z_0}f(z)=\infty$.
\end{theorem}
\begin{proof}
\begin{enumerate}
    \item Пусть $z_0$ – полюс $f(z)$, тогда в некоторой проколотой окрестности K точки z0 имеется представление $f(z)=\frac{c_{-k}}{(z-z_0)^k}+...+c_0+c_1(z-z_0)+...(3)$,
    где $c_{-k}\neq0$. Равенство (3) можно переписать в виде:
    $f(z)(z-z_0)^k=c_{-k}+c_{-k+1}(z-z_0)+...+c_0(z-z_0)^k+...$, причем ряд, стоящий в правой части последнего равенства, сходится в некотором круге $K_r=\{z:|z-z_0|<r\}$. Если $\phi(z)$ сумма этого ряда, то $\phi(z) \in A(K_r)$, $\phi(z_0)=c_{-k}\neq0$. Поэтому $f(z)=\frac{\phi(z)}{(z-z_0)^k}$ и очевидно $\displaystyle\lim_{z\rightarrow z_0}f(z)=\infty$.
    \item Обратно, пусть $\displaystyle\lim_{z\rightarrow z_0}f(z)=\infty$. Тогда существует
    проколотая окрестность $K$ точки $z_0$, где $f(z) \neq 0$, поэтому
    в $K$ определена аналитическая функция $g(z) = \frac{1}{f(z)}$, причём справедливо представление: $g(z)=a_k(z-z_0)^k+a_{k+1}(z-z_0)^{k+1}+...=(z-z_0)^k(a_k+a_{k+1}(z-z_0)+...)$, где $k\geq1$, $a_k\neq0$. Значит $g(z)=(z-z_0)^k\phi(z)$, где $\phi(z_0)\neq0$. Тогда $f(z)=\frac{1}{g(z)}=\frac{1}{(z-z_0)^k}\frac{1}{\phi(z)}=\{\phi(z) \in A(K) \Rightarrow \frac{1}{\phi(z)} \in A(K) \text{ значит можно разложить в ряд Лорана}\}=
    \frac{1}{(z-z_0)^k}(b_0+b_1(z-z_0)+...)$, где $b_0=\frac{1}{\phi(z_0)}=\frac{1}{a_k}\neq0$, т.e $z_0$ -- полюс $f(z)$.
\end{enumerate}
\end{proof}
\begin{theorem}
   Точка $z_0$ – полюс порядка $k$ функции $f(z)$ тогда и только тогда, когда в K справедливо
   представление: $f(z)=\frac{\phi(z)}{(z-z_0)^k}$, где $\phi(z)\in A(z_0)$, $\phi(z_0)\neq 0$.
\end{theorem}
\begin{proof}
    Утверждение следует из доказательства предыдущей теоремы
\end{proof}
\begin{sle}
Для того, чтобы в точке $z0$ был полюс
порядка $k$ функции $f(z)$, необходимо и достаточно, чтобы функция $g(z)=\begin{cases}
    \frac{1}{f(z)} \text{ ,где } z\neq z_0\\
    0\text{ ,где } z=z_0
   \end{cases}$ имела в точке $z_0$ нуль порядка $k$.
\end{sle}
\begin{proof}
    Точка $z0$ – нуль порядка $k$ функции $g(z)$ тогда и только тогда, когда $g(z)=(z-z_0)^kg_1(z)$, где $g_1(z) \in A(z_0)$, $g_1(z_0)\neq0$. Далее по теореме.
\end{proof}
\begin{theorem}
   Изолированная особая точка $z_0$ функции
    $f(z)$ является существенно особой тогда и только тогда,
    когда не существует $\displaystyle\lim_{z\rightarrow z_0}f(z)$.
\end{theorem}

\begin{theorem}[Сохоцкого–Казорати–Вейерштрасса]
Пусть $z_0$ – существенно особая точка функции $f(z)$. Тогда
для произвольного числа $A \in \mathbb{C}$ найдётся такая последовательность ${z_n}$, сходящаяся к $z_0$, что $f(z_n)\rightarrow A$, $n \rightarrow \infty$.
\end{theorem}
\begin{proof}
    Для заданного числа $A$ такую последовательность $z_n$, будем называть $A$-последовательностью Сохоцкого. В любой окрестности сущ.особой точки $z_0$ $f(z)$ не может быть ограничена(иначе была бы устранимой по теореме 2)$\Longrightarrow$ найдется последовательность точек $z_n^{'}: |z_n^{'}-z_0|<\frac{1}{n} \text{ и } f(z_n^{'}) > n.$ Эту послед можно принять за $\infty$-последовательность Сохоцкого. $\exists$-ние $A$-последовательности Сохоцкого: докажем от противного: Пусть такой послед не $\exists \Longrightarrow$ найдутся окрестности $U_\delta: 0<|z_-z_0|<\delta$ и $\alpha>0: |f(z)-A|>\alpha$, $\forall z \in U_\delta(1)$. Положим $\phi(z)=\frac{1}{f(z)-A}$. Функция $\phi(z)$ определена и $\phi(z) \in A(U_\delta)$. Из $(1)\Longrightarrow |\phi(z)|<\frac{1}{\alpha} \forall z \in U_\delta$. По теореме.2 $z_0$ устранимая точка для $\phi(z)\Longrightarrow \exists \displaystyle\lim_{z \rightarrow z_0}\phi(z)$. Найдем его значения $c_0$ с помощью $\infty$-последовательности Сохоцкого $z_n^{'}: c_0 = \displaystyle \lim_{n \rightarrow \infty} \frac{1}{(f(z_n^{'})-A)}=0 \Longrightarrow \displaystyle \lim_{z \rightarrow z_0} \frac{1}{(f(z)-A)}=0$, что возможно лишь при $\displaystyle\lim_{z\rightarrow z_0}f(z)=\infty \Longrightarrow z_0$ -- полюс $f(z)$, противоречие. 
\end{proof}

\begin{theorem}[Пикара]
Пусть $a \in C$ – существенно особая точка для $f(z)$. Тогда в любой проколотой окрестности
точки $a$, $f(z)$ принимает все комплексные значения, причём каждое бесконечное число раз
(за исключением, быть может, одного $A$)
\end{theorem}

\end{document}
